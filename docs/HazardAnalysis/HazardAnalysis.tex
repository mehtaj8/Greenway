\documentclass{article}

\usepackage{booktabs}
\usepackage{tabularx}

\title{Hazard Analysis\\\progname}

\author{\authname}

\date{}

%% Comments

\usepackage{color}

\newif\ifcomments\commentstrue %displays comments
%\newif\ifcomments\commentsfalse %so that comments do not display

\ifcomments
\newcommand{\authornote}[3]{\textcolor{#1}{[#3 ---#2]}}
\newcommand{\todo}[1]{\textcolor{red}{[TODO: #1]}}
\else
\newcommand{\authornote}[3]{}
\newcommand{\todo}[1]{}
\fi

\newcommand{\wss}[1]{\authornote{blue}{SS}{#1}} 
\newcommand{\plt}[1]{\authornote{magenta}{TPLT}{#1}} %For explanation of the template
\newcommand{\an}[1]{\authornote{cyan}{Author}{#1}}

%% Common Parts

\newcommand{\progname}{ProgName} % PUT YOUR PROGRAM NAME HERE
\newcommand{\authname}{Team \#, Team Name
\\ Student 1 name and macid
\\ Student 2 name and macid
\\ Student 3 name and macid
\\ Student 4 name and macid} % AUTHOR NAMES                  

\usepackage{hyperref}
    \hypersetup{colorlinks=true, linkcolor=blue, citecolor=blue, filecolor=blue,
                urlcolor=blue, unicode=false}
    \urlstyle{same}
                                


\begin{document}

\maketitle

\newpage

\section*{Revision History}

\begin{tabularx}{\textwidth}{p{3cm}p{2cm}X}
\toprule {\bf Date} & {\bf Version} & {\bf Notes}\\
\midrule
19-10-2022 & 1.0 & Hazard Analysis Version 1\\
\bottomrule
\end{tabularx}

\newpage

\tableofcontents

\newpage

\section{Introduction}
\subsection{Background}
\subsubsection{Scope}
The scope of the proposed software, Greenway, is a mapping software that not only 
gives fuel efficient directions to the intended destination but provides the user with fuel cost calculations.
The intention is to calculate fuel costs using gas price data, car mileage information and terrain information
to show how much money it will take to get to the intended destination using the most fuel efficient route.

\subsubsection{Document Purpose}
The purpose of this document is to identify potential hazardous components of Greenway
and methods to mitigate these risks to an acceptable level. The following sections describe
details of potential hazards in depth for each subsystem which also helps to understand how
the system works as a whole to avoid these hazards.

\subsection{Overview}
\subsubsection{Scope of Hazard Analysis}
The scope of the this document covers all individual components that build up the entire system. These components include the "Car Selection System", the "Destination Selection System", the "Route Calculation System", the "Terrain Dat Collection System", the "Gas Price Collection System", and the "Fuel Cost Calculation System". The components mentioned each have their own safety considerations.

\subsubsection{Definition of Hazard}
Team Roadkill has defined their definition of hazard to be, " ".


\section{Component Overview}
Below describes each of the subsystems that make up the larger system.

\subsection{Car Selection System}
This component is responsible for allowing the user to input car data and collecting car information, such as mileage.

\subsection{Destination Selection System}
This component lets the user input the desired destination.

\subsection{Route Calculation System}
This component is responsible for calculating the most fuel efficient route to the destionation selected by the user.

\subsection{Terrain Data Collection System}
This component is responsible for collecting terrain data and providing it to the fuel cost calculation component.

\subsection{Gas Price Collection System}
This component is responsible for collecting gas price data from local gas stations.

\subsection{Fuel Cost Calculation System}
This component is responsible for calculating fuel cost taking into consideration fuel mileage data, terrain data, and gas price data.

\section{Safety Considerations}
\begin{itemize}
	\item[\textbf{Issue 1:}]
	\item[\textbf{Solution 1:}]
	\item[\textbf{Issue 2:}]
	\item[\textbf{Solution 2:}]
\end{itemize}

\subsection{Car Selection System}


\subsection{Destination Selection System}

\subsection{Route Calculation System}

\subsection{Terrain Data Collection System}

\subsection{Fuel Cost Calculation System}

\section{FMEA Worksheet}

\section{Conclusion}
In summary, it is important to realize the potential hazards of each function to avoid potential risks and negative experiences for the user. The above document contents highlight
important safety considerations along with risks/failures for each function. This will ensure
the program is developed with expected behaviour.

\end{document}