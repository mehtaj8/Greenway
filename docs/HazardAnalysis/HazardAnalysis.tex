\documentclass{article}

\usepackage{booktabs}
\usepackage{tabularx}

\title{Hazard Analysis\\\progname}

\author{\authname}

\date{}

%% Comments

\usepackage{color}

\newif\ifcomments\commentstrue %displays comments
%\newif\ifcomments\commentsfalse %so that comments do not display

\ifcomments
\newcommand{\authornote}[3]{\textcolor{#1}{[#3 ---#2]}}
\newcommand{\todo}[1]{\textcolor{red}{[TODO: #1]}}
\else
\newcommand{\authornote}[3]{}
\newcommand{\todo}[1]{}
\fi

\newcommand{\wss}[1]{\authornote{blue}{SS}{#1}} 
\newcommand{\plt}[1]{\authornote{magenta}{TPLT}{#1}} %For explanation of the template
\newcommand{\an}[1]{\authornote{cyan}{Author}{#1}}

%% Common Parts

\newcommand{\progname}{ProgName} % PUT YOUR PROGRAM NAME HERE
\newcommand{\authname}{Team \#, Team Name
\\ Student 1 name and macid
\\ Student 2 name and macid
\\ Student 3 name and macid
\\ Student 4 name and macid} % AUTHOR NAMES                  

\usepackage{hyperref}
    \hypersetup{colorlinks=true, linkcolor=blue, citecolor=blue, filecolor=blue,
                urlcolor=blue, unicode=false}
    \urlstyle{same}
                                


\begin{document}

\maketitle

\newpage

\section*{Revision History}

\begin{tabularx}{\textwidth}{p{3cm}p{2cm}X}
\toprule {\bf Date} & {\bf Version} & {\bf Notes}\\
\midrule
19-10-2022 & 1.0 & Hazard Analysis Version 1\\
\bottomrule
\end{tabularx}

\newpage

\tableofcontents

\newpage

\section{Introduction}
\subsection{Background}
\subsubsection{Scope}
The scope of the proposed software, Greenway, is a mapping software that not only 
gives fuel efficient directions to the intended destination but provides the user with fuel cost calculations.
The intention is to calculate fuel costs using gas price data, car mileage information and terrain information
to show how much money it will take to get to the intended destination using the most fuel efficient route.

\subsubsection{Document Purpose}
The purpose of this document is to identify potential hazardous components of Greenway
and methods to mitigate these risks to an acceptable level. The following sections describe
details of potential hazards in depth for each subsystem which also helps to understand how
the system works as a whole to avoid these hazards.

\subsection{Overview}
\subsubsection{Scope of Hazard Analysis}
The scope of the this document covers all individual components that build up the entire system. These components include the "Car Selection System", the "Destination Selection System", the "Route Calculation System", the "Terrain Data Collection System", the "Gas Price Collection System", and the "Fuel Cost Calculation System". The components mentioned each have their own safety considerations.

\subsubsection{Definition of Hazard}
Team Roadkill has defined hazard to be, "any aspect, property, or feature of Greenway which communicates incorrect information to the user or negatively impacts the user experience.".


\section{Component Overview}
The following is a description of each of Greenway's subsystems that make up the larger system.

\subsection{Car Selection System}
This component is responsible for allowing the user to input car data and collecting relevant car information, such as mileage.

\subsection{Destination Selection System}
This component lets the user input the desired destination.

\subsection{Route Calculation System}
This component is responsible for calculating the most fuel efficient route to the destionation selected by the user.

\subsection{Terrain Data Collection System}
This component is responsible for collecting terrain data and providing it to the fuel cost calculation component.

\subsection{Gas Price Collection System}
This component is responsible for collecting gas price data from local gas stations.

\subsection{Fuel Cost Calculation System}
This component is responsible for calculating fuel cost taking into consideration fuel mileage data, terrain data, and gas price data.

\section{Safety Considerations}

\subsection{Car Selection System}
\begin{itemize}
	\item[\textbf{Issue 1:}] Incorrect input
	\item[\textbf{Solution 1:}] The system will allow the user to change the selected car. It will also allow the user to verify their selection 
	before confirming it as there selected car.
	\item[\textbf{Issue 2:}] Invalid input
	\item[\textbf{Solution 2:}] The system will check if their car exists in the database, and will let the user enter a new car if their current selection
	isn't in the database.
	\item[\textbf{Issue 3:}] Invalid data
	\item[\textbf{Solution 3:}] The system will allow the user to verify if our estimate is sensible for their car. If it is not then the user can 
	manually enter required data like mileage. 
	\item[\textbf{Issue 4:}] No internet connection
	\item[\textbf{Solution 4:}] The system will prevent user input and indicate to the user that Internet connectivity is required for the functionality of the system.
\end{itemize}

\subsection{Destination Selection System}
\begin{itemize}
	\item[\textbf{Issue 1:}] Incorrect input
	\item[\textbf{Solution 1:}] The system will allow the user to change the selected destination. It will also allow the user to verify their pick 
	before confirming it as there selected destination.
	\item[\textbf{Issue 2:}] Invalid input
	\item[\textbf{Solution 2:}] The system will check if their destination is valid (reachable by car and exists on the map) before letting them set it as their desired destination.
	\item[\textbf{Issue 3:}] No internet connection
	\item[\textbf{Solution 3:}] The system will prevent user input and indicate to the user that Internet connectivity is required for the functionality of the system.
\end{itemize}

\subsection{Route Calculation System}
\begin{itemize}
	\item[\textbf{Issue 1:}] Invalid output
	\item[\textbf{Solution 1:}] The system will validate the output with estimation tools built into to it ensure that the route is a acceptable range.
	\item[\textbf{Issue 2:}] No internet connection
	\item[\textbf{Solution 2:}] The system will prevent user input and indicate to the user that Internet connectivity is required for the functionality of the system.
	\item[\textbf{Issue 3:}] Hidden Costs
	\item[\textbf{Solution 3:}] The system will ensure that the user know only fuel efficiency is being used to calculate the given route and it does 
	not necessarily ensure the cheapest route to take.
\end{itemize}

\subsection{Terrain Data Collection System}
\begin{itemize}
	\item[\textbf{Issue 1:}] Invalid output
	\item[\textbf{Solution 1:}] The system will validate the output against an external database to ensure that the data is in an acceptable range.
	\item[\textbf{Issue 2:}] No internet connection
	\item[\textbf{Solution 2:}] The system will prevent user input and indicate to the user that Internet connectivity is required for the functionality of the system.
	\item[\textbf{Issue 3:}] Extreme Weather
	\item[\textbf{Solution 3:}] The system will ensure that the user knows weather conditions can make certain routes undesirable, and that they 
	should do research before going on specific routes.
\end{itemize}

\subsection{Fuel Cost Calculation System}
\begin{itemize}
	\item[\textbf{Issue 1:}] Invalid output
	\item[\textbf{Solution 1:}] The system will validate the output against an external database to ensure that the data is in a acceptable range.
	\item[\textbf{Issue 2:}] No internet connection
	\item[\textbf{Solution 2:}] The system will prevent user input and indicate to the user that Internet connectivity is required for the functionality of the system.
	\item[\textbf{Issue 3:}] Fuel Cost changes
	\item[\textbf{Solution 3:}] The system will ensure that the user knows when the application checked the fuel prices and that there might be certain inaccuracies 
	with the estimation due to prices changing constantly.
\end{itemize}

\section{FMEA Worksheet}

\section{Conclusion}
In summary, it is important to realize the potential hazards of each function to avoid potential risks and negative experiences for the user. The above document contents highlight
important safety considerations along with risks/failures for each function. This will ensure
the program is developed with expected behaviour.

\end{document}