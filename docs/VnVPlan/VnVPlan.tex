\documentclass[12pt, titlepage]{article}

\usepackage{booktabs}
\usepackage{tabularx}
\usepackage{hyperref}
\usepackage{float}
\hypersetup{
    colorlinks,
    citecolor=blue,
    filecolor=black,
    linkcolor=red,
    urlcolor=blue
}
\usepackage[round]{natbib}

%% Comments

\usepackage{color}

\newif\ifcomments\commentstrue %displays comments
%\newif\ifcomments\commentsfalse %so that comments do not display

\ifcomments
\newcommand{\authornote}[3]{\textcolor{#1}{[#3 ---#2]}}
\newcommand{\todo}[1]{\textcolor{red}{[TODO: #1]}}
\else
\newcommand{\authornote}[3]{}
\newcommand{\todo}[1]{}
\fi

\newcommand{\wss}[1]{\authornote{blue}{SS}{#1}} 
\newcommand{\plt}[1]{\authornote{magenta}{TPLT}{#1}} %For explanation of the template
\newcommand{\an}[1]{\authornote{cyan}{Author}{#1}}

%% Common Parts

\newcommand{\progname}{ProgName} % PUT YOUR PROGRAM NAME HERE
\newcommand{\authname}{Team \#, Team Name
\\ Student 1 name and macid
\\ Student 2 name and macid
\\ Student 3 name and macid
\\ Student 4 name and macid} % AUTHOR NAMES                  

\usepackage{hyperref}
    \hypersetup{colorlinks=true, linkcolor=blue, citecolor=blue, filecolor=blue,
                urlcolor=blue, unicode=false}
    \urlstyle{same}
                                


\begin{document}

\title{Project Title: System Verification and Validation Plan for \progname{}} 
\author{Author Name}
\date{\today}
	
\maketitle

\pagenumbering{roman}

\section{Revision History}

\begin{tabularx}{\textwidth}{p{3cm}p{2cm}X}
\toprule {\bf Date} & {\bf Version} & {\bf Notes}\\
\midrule
Date 1 & 1.0 & Notes\\
Date 2 & 1.1 & Notes\\
\bottomrule
\end{tabularx}

\newpage

\tableofcontents

\listoftables
\wss{Remove this section if it isn't needed}

\listoffigures
\wss{Remove this section if it isn't needed}

\newpage

\section{Symbols, Abbreviations and Acronyms}

\renewcommand{\arraystretch}{1.2}
\begin{tabular}{l l} 
  \toprule		
  \textbf{symbol} & \textbf{description}\\
  \midrule 
  T & Test\\
  \bottomrule
\end{tabular}\\

\wss{symbols, abbreviations or acronyms -- you can simply reference the SRS
  \citep{SRS} tables, if appropriate}

\newpage

\pagenumbering{arabic}

This document ... \wss{provide an introductory blurb and roadmap of the
  Verification and Validation plan}

\section{General Information}

\subsection{Summary}


The goal of Greenway is to create an application that allows users to provide car information
and a destination, in return the application supplies the amount of money it will take for
the user to reach that destination using the most fuel-efficient route calculated by the app.
The application uses real-time gas price data, information on mileage data, and terrain data
to calculate and provide the amount of money it would take to reach a destination for users
who drive many different types of gas vehicles; this allows them to use this app in many
ways, such as, figuring out which gym may provide the most bang for their buck based on
distance from their location.

\subsection{Objectives}

The objective of the document is to test the various subsystems identified in the system requirements and the systems design documents to ensure correctness and demonstrate adequate usability. 
Often times software may have
bugs that is experienced by the end-user. This document will try to mitigate those issues by
ensuring the underlying logic for the subsystems are performing as expected. This will be
done through a suite of Unit Tests to test the functional and non-functional requirements.
The code tested will be the underlying logic functions that interact with the database and
send information back to the client since the rest of the system is front-end UI aspects.


\subsection{Relevant Documentation}


The relevant documents include:
\begin{itemize}
\item Software Requirment Specification Document
\item Systems Design Document
\item Hazard Analysis Document
\end{itemize}


\section{Plan}

This section includes the following sections, "Verification and Validation Team", "SRS Verification Plan", "Design Verification Plan", "Implementation Verfication Plan", "Automated Testing and Verification Tools", "Software Validation Plan".

\subsection{Verification and Validation Team}

The following project members are responsible for all procedures of the verification and validation processes including writing and executing tests:
\begin{itemize}
    \item Utsharga Rozario
    \item Jash Mehta
    \item Priyansh Shah
    \item Bilal Shaikh
    \item Sharjil Mohsin
    \item Pranay Kotian
\end{itemize}

\subsection{SRS Verification Plan}

The following plans indicate what our team intends to do for SRS verification:
\begin{itemize}
    \item \textbf{Review by teammates:} This plan involves going through each SRS and verifying whether each SRS is still valid within our given scope.
    \item \textbf{Review by stakeholders:} This plan involves going through each SRS with our stakeholder and how they view them in perspective to your usage.
    \item \textbf{Task based inspection:} This plan involves going through each SRS as a task, generally done by a teammate or stakeholder.
    \item \textbf{Checklist:} This plan involves verifying if each SRS has been met through our previously set checklist in our SRS document.
\end{itemize}

\subsection{Design Verification Plan}

The following plans indicate what our team intends to do for Design verification:
\begin{itemize}
    \item \textbf{Review by teammates:} This plan involves going through the design of the project with our teammates through a high-fidelity prototype or a functional prototype, and verifying if the design is meets the expectation stated in the SRS.
    \item \textbf{Review by stakeholders:} This plan involves going through the design of the project with our stakeholders through a high-fidelity prototype or a functional prototype, and verifying if the design is meets the expectation stated in the SRS.
    \item \textbf{Task based inspection:} This plan involves going through each feature as a task in a high-fidelity prototype or a functional prototype, generally done by a teammate or stakeholder.
    \item \textbf{Checklist:} This plan involves verifying if each design requirements have been met through our previously set checklist in our SRS document.
\end{itemize}

\subsection{Implementation Verification Plan}
The following plans indicate what our team intends to do for Implementation verification:
\begin{itemize}
    \item \textbf{Code Inspection:} This will be used for the test plans in section \ref{PIT}.
    \item \textbf{Static Analysis:} This will be used for the test plans in section \ref{MI} and section \ref{BE}.
    \item \textbf{Non-functional Testing:} Non-functional Requirements test plans are written in details in section \ref{NFR}.
\end{itemize}

\subsection{Automated Testing and Verification Tools}
The following Automated Testing Tools:
\begin{itemize}
    \item \textbf{Cypress:} Cypress is an automated testing tool from front-end development and has ability to take screenshots and video recordings of tests, allows developers to do asynchronous testing, and has In-browser debugging environment. Hence will be used to perform front-end testing on React.
    \item \textbf{Mocha:} Mocha is one of the oldest and most well-known testing frameworks for Node.js and hence will be used for our project. It has also evolved with Node.js and the JavaScript language over the years, such as supporting callbacks, promises, and async/await which we intend to use in our back-end.
    \item \textbf{Mongo Orchestration:} Mongo Orchestration is an HTTP server providing a RESTful interface to MongoDB process management running on the same machine and will be used to test our MongoDB database.
\end{itemize}
The following Verification Tools:
\begin{itemize}
    \item \textbf{EsLint:} ESLint is a tool for identifying and reporting on patterns found in ECMAScript/JavaScript code, with the goal of making code more consistent and avoiding bugs and hence will used for linting our front-end (React) and back-end (Node.js).
\end{itemize}

\subsection{Software Validation Plan}

There are no available external data that can help validate the software. Hence, there are no plans for validation for now.

\section{System Test Description}
	
\subsection{Tests for Functional Requirements}

The areas of testing listed below are the following: Start screen, Map Interactions, 
Backend processing, Results Display. Start screen covers functional requirements 1-5 
from the SRS as all those requirements are related to the inputs and functionality that 
exists in the program before any processing or output is shown. Map Interaction covers 
functional requirements 6-8 as these requirements are concerned with functionality related 
to the map on display. Backend processing covers functional requirements 9-12 as these 
requirements are concerned with calculations and, external data collection and processing 
alike.

\subsubsection{Start Screen}

Start screen covers functional requirements 1-5 from the SRS as all those requirements 
are related to the inputs and functionality that exists in the program before any 
processing or output is shown
		
\paragraph{Preliminary Information Tests} \label{PIT}

\begin{enumerate}

\item{preliminary-information-test-1\\}

Control: Automatic
					
Initial State: No input in the start screen
					
Input: Start Location
					
Output: Status Message stating if the Location was accepted

Test Case Derivation: The status message should be accepted the location as FR1 in the SRS 
requires the system to be able to take that as an input.
					
How test will be performed: The testing framework will use a valid start location to input 
it into the field and ensure the status message is accepted the location.
					
\item{preliminary-information-test-2\\}

Control: Automatic
					
Initial State: No input in the start screen
					
Input: Location of the Destination
					
Output: Status Message stating if the Location was accepted

Test Case Derivation: The status message should be accepted the location as FR2 in the SRS 
requires the system to be able to take that as an input.
					
How test will be performed: The testing framework will use a valid location for a destination to input 
it into the field and ensure the status message is accepted the location.

\item{preliminary-information-test-3\\}

Control: Automatic
					
Initial State: No input in the start screen
					
Input: Car Details
					
Output: Status Message stating if the Car Information was accepted

Test Case Derivation: The status message should be accepted the Car Information as FR3 in the SRS 
requires the system to be able to take that as an input.
					
How test will be performed: The testing framework will use valid Car details for details to input 
it into the field and ensure the status message is accepted the Car Information.

\item{preliminary-information-test-4\\}

Control: Automatic
					
Initial State: Accepted Car details in the start screen
					
Input: Car mileage/fuel economy information
					
Output: Status Message stating if the Car mileage/fuel economy information was accepted

Test Case Derivation: The status message should be accepted the Car mileage/fuel economy information as FR4 in the SRS 
requires the system to be able to take that as an input.
					
How test will be performed: The testing framework will use valid Car mileage/fuel economy information for details to input 
it into the field and ensure the status message is accepted the Car mileage/fuel economy information.

\item{preliminary-information-test-5\\}

Control: Automatic
					
Initial State: Accepted Car details in the start screen
					
Input: Car information
					
Output: Status Message stating if the Car information was updated

Test Case Derivation: The status message should be accepted the Car information as FR5 in the SRS 
requires the system to be able to take that as an input.
					
How test will be performed: The testing framework will use valid Car information for details to input 
it into the field and ensure the status message is updated the Car information.

\end{enumerate}

\subsubsection{Map Interactions} \label{MI}

Map Interaction covers functional requirements 6-8 as these requirements 
are concerned with functionality related to the map on display.
		
\paragraph{Map Tests}

\begin{enumerate}

\item{map-test-1\\}

Control: Manual
					
Initial State: Start Screen finished
					
Input: None
					
Output: If the Map displays a route from start to end based on start screen input.

Test Case Derivation: The Map should display a route from start to end as specified in the Start screen, as FR6 States.
					
How test will be performed: The tester will ensure that the map has the correct route showing on it for a given start and 
end destination with a correct route to compare with.
					
\item{map-test-2\\}

Control: Manual
					
Initial State: Start Screen finished
					
Input: None
					
Output: If the Map displays all gas stations that it can possibly encounter along the route from a start to end destination.

Test Case Derivation: The Map should display all gas stations along the route from start to end as specified in the Start screen, as FR7 States.
					
How test will be performed: The tester will ensure that the map has all the gas stations along the route it shows and that 
they are being correctly displayed with a map to reference.
					
\item{map-test-3\\}

Control: Manual
					
Initial State: Start Screen finished
					
Input: None
					
Output: If correct gas prices come up when gas stations along the displayed route are clicked upon.

Test Case Derivation: The Map should display all gas prices in the gas stations along the route from start to end as specified in the Start screen, as FR8 States.
					
How test will be performed: The tester will click on all gas stations along the route and check that they are 
displaying gas prices in a range given beforehand to the tester.

\end{enumerate}

\subsubsection{Backend Processing} \label{BE}

Backend processing covers functional requirements 9-12 as these 
requirements are concerned with calculations and, external data collection and processing 
alike.
		
\paragraph{Backend Tests}

\begin{enumerate}

\item{backend-test-1\\}

Control: Automatic
					
Initial State: Start Screen finished
					
Input: Route Details
					
Output: If the route returned to the user is one that is optimizing the distance and elevation perfectly to reduce fuel costs.

Test Case Derivation: The route displayed to the user should be the most fuel efficient route, as FR9 States.
					
How test will be performed: The testing framework will have correct nodes pertaining to the most fuel efficient route and will 
compare them with nodes outputted by the back end to ensure that they are correct.
					
\item{backend-test-2\\}

Control: Automatic
					
Initial State: Start Screen finished
					
Input: Route Details
					
Output: If the database call on the backend returns correct elevation data for a route.

Test Case Derivation: The most fuel efficient needs accurate elevation data and as such the FR10 is derived from the needs of FR9. And 
the test is designed to ensure that FR10 is fulfilled in the design.
					
How test will be performed: The testing framework will test that the backend is getting the correct elevation data from the database 
with a reference dataset with correct elevation data.
					
\item{backend-test-3\\}

Control: Automatic
					
Initial State: Start Screen finished
					
Input: Route Details
					
Output: If correct fuel consumption is given by backend for different parts of the route which have different elevation metrics.

Test Case Derivation: The application should be able to calculate accurate fuel consumption in different elevation conditions as 
distance is not the only metric for determining the most fuel efficient route. This test is made for FR11.
					
How test will be performed: The testing framework will test that the backend is calculating the correct fuel consumption data from the database 
with a reference dataset with correct fuel consumption data.
					
\item{backend-test-4\\}

Control: Automatic
					
Initial State: Start Screen finished
					
Input: Route Details
					
Output: If correct total cost is calculated by the backend for a certain route with a start and end decisions as the route details.

Test Case Derivation: The application should be able to calculate accurate total cost for a certain route, as FR12 states.
					
How test will be performed: The testing framework will test that the backend is to ensure that the a correct total cost is being returned 
with a correct total cost for a route as reference.

\end{enumerate}

\subsection{Tests for Nonfunctional Requirements} \label{NFR}

\wss{The nonfunctional requirements for accuracy will likely just reference the
  appropriate functional tests from above.  The test cases should mention
  reporting the relative error for these tests.}

\wss{Tests related to usability could include conducting a usability test and
  survey.}

\subsubsection{Area of Testing1}
		
\paragraph{Title for Test}

\begin{enumerate}

\item{test-id1\\}

Type: 
					
Initial State: 
					
Input/Condition: 
					
Output/Result: 
					
How test will be performed: 
					
\item{test-id2\\}

Type: Functional, Dynamic, Manual, Static etc.
					
Initial State: 
					
Input: 
					
Output: 
					
How test will be performed: 

\end{enumerate}

\subsubsection{Area of Testing2}

...

\newpage

\subsection{Traceability Between Test Cases and Requirements}
  
  \begin{table}[!hbp]
	
	\begin{tabular}{|p{3.5cm}|p{6.5cm}|p{4.5cm}|}

	\hline
	\textbf{Requirement \#} & \textbf{Description}                                                                                                                                                                  & \textbf{Test ID(s)}                                                                                          \\ \hline
	FR1                     & The system shall allow the user a way to input a start location.                                                                                     & preliminary-information-test-1                                                                                                 \\ \hline
	FR2                     & The system shall allow the user a way to input a final destination.                                                                             & preliminary-information-test-2                                                                                               \\ \hline
	FR3                     & The system shall allow the user a way to select any car model and make.                                                                                                  & preliminary-information-test-3                                                                                         \\ \hline
	FR4                     & The system shall allow the user a way to input the fuel economy of their car.  & preliminary-information-test-4                                                                                         \\ \hline
	FR5                     & The system shall have a way to update available car models from either user input or external sources.                                                                                                         & preliminary-information-test-5                                                                                          \\ \hline
	FR6                     & The system shall have a map to show journey from start to final destination.                                                                                                                                  & map-test-1                                                                                          \\ \hline
	FR7                     & The system shall display the gas stations along the route.                                                                                                         & map-test-2                                                                                          \\ \hline
	FR8                    & The system shall display gas prices at stations on route to destination.                                                                                                         & map-test-3                                                                                          \\ \hline
	FR9                     & The system shall have a way to calculate the most fuel efficient route.                                                                                                         & backend-test-1                                                                                          \\ \hline
	FR10                    & The system shall have a way to collect elevation data along the suggested route.                                                                                                         & backend-test-2                                                                                          \\ \hline
	FR11                    & The system shall have a way to calculate fuel consumption on different terrain elevations.                                                                                                         & backend-test-3                                                                                          \\ \hline
	FR12                     & The system must calculate the total cost of the journey.                                                                                                         & backend-test-4                                                                                          \\ \hline
	
	\end{tabular}
	
  \end{table}

\newpage

\section{Unit Test Description}

\wss{Reference your MIS and explain your overall philosophy for test case
  selection.}  
\wss{This section should not be filled in until after the MIS has
  been completed.}

\subsection{Unit Testing Scope}

\wss{What modules are outside of the scope.  If there are modules that are
  developed by someone else, then you would say here if you aren't planning on
  verifying them.  There may also be modules that are part of your software, but
  have a lower priority for verification than others.  If this is the case,
  explain your rationale for the ranking of module importance.}

\subsection{Tests for Functional Requirements}

\wss{Most of the verification will be through automated unit testing.  If
  appropriate specific modules can be verified by a non-testing based
  technique.  That can also be documented in this section.}

\subsubsection{Module 1}

\wss{Include a blurb here to explain why the subsections below cover the module.
  References to the MIS would be good.  You will want tests from a black box
  perspective and from a white box perspective.  Explain to the reader how the
  tests were selected.}

\begin{enumerate}

\item{test-id1\\}

Type: \wss{Functional, Dynamic, Manual, Automatic, Static etc. Most will
  be automatic}
					
Initial State: 
					
Input: 
					
Output: \wss{The expected result for the given inputs}

Test Case Derivation: \wss{Justify the expected value given in the Output field}

How test will be performed: 
					
\item{test-id2\\}

Type: \wss{Functional, Dynamic, Manual, Automatic, Static etc. Most will
  be automatic}
					
Initial State: 
					
Input: 
					
Output: \wss{The expected result for the given inputs}

Test Case Derivation: \wss{Justify the expected value given in the Output field}

How test will be performed: 

\item{...\\}
    
\end{enumerate}

\subsubsection{Module 2}

...

\subsection{Tests for Nonfunctional Requirements}

\wss{If there is a module that needs to be independently assessed for
  performance, those test cases can go here.  In some projects, planning for
  nonfunctional tests of units will not be that relevant.}

\wss{These tests may involve collecting performance data from previously
  mentioned functional tests.}

\subsubsection{Module ?}
		
\begin{enumerate}

\item{test-id1\\}

Type: \wss{Functional, Dynamic, Manual, Automatic, Static etc. Most will
  be automatic}
					
Initial State: 
					
Input/Condition: 
					
Output/Result: 
					
How test will be performed: 
					
\item{test-id2\\}

Type: Functional, Dynamic, Manual, Static etc.
					
Initial State: 
					
Input: 
					
Output: 
					
How test will be performed: 

\end{enumerate}

\subsubsection{Module ?}

...

\subsection{Traceability Between Test Cases and Modules}

\wss{Provide evidence that all of the modules have been considered.}
				
\bibliographystyle{plainnat}

\bibliography{../../refs/References}

\newpage

\section{Appendix}

This is where you can place additional information.

\subsection{Symbolic Parameters}

The definition of the test cases will call for SYMBOLIC\_CONSTANTS.
Their values are defined in this section for easy maintenance.

\subsection{Usability Survey Questions?}

\wss{This is a section that would be appropriate for some projects.}

\newpage{}
\section*{Appendix --- Reflection}

The information in this section will be used to evaluate the team members on the
graduate attribute of Lifelong Learning.  Please answer the following questions:

\begin{enumerate}
  \item 
  \item 
\end{enumerate}

\end{document}