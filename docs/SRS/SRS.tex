\documentclass[12pt]{article}

\usepackage{amsmath, mathtools}
\usepackage{amsfonts}
\usepackage{amssymb}
\usepackage{graphicx}
\usepackage{colortbl}
\usepackage{xr}
\usepackage{hyperref}
\usepackage{longtable}
\usepackage{xfrac}
\usepackage{tabularx}
\usepackage{float}
\usepackage{siunitx}
\usepackage{booktabs}
\usepackage{caption}
\usepackage{pdflscape}
\usepackage{afterpage}

\usepackage[round]{natbib}

%\usepackage{refcheck}

\hypersetup{
    bookmarks=true,         % show bookmarks bar?
      colorlinks=true,       % false: boxed links; true: colored links
    linkcolor=red,          % color of internal links (change box color with linkbordercolor)
    citecolor=green,        % color of links to bibliography
    filecolor=magenta,      % color of file links
    urlcolor=cyan           % color of external links
}

\input{../Comments}
%% Common Parts

\newcommand{\progname}{Greenway} % PUT YOUR PROGRAM NAME HERE
\newcommand{\authname}{Team \#11, Roadkill
\\ Priyansh Shah, shahp36
\\ Utsharga Rozario, rozariou
\\ Jash Mehta, mehtaj8
\\ Bilal Shaikh, shaikb2
\\ Pranay Kotian, kotianp
\\ Sharjil Mohsin, mohsis2} % AUTHOR NAMES                  

\usepackage{hyperref}
    \hypersetup{colorlinks=true, linkcolor=blue, citecolor=blue, filecolor=blue,
                urlcolor=blue, unicode=false}
    \urlstyle{same}
                                


% For easy change of table widths
\newcommand{\colZwidth}{1.0\textwidth}
\newcommand{\colAwidth}{0.13\textwidth}
\newcommand{\colBwidth}{0.82\textwidth}
\newcommand{\colCwidth}{0.1\textwidth}
\newcommand{\colDwidth}{0.05\textwidth}
\newcommand{\colEwidth}{0.8\textwidth}
\newcommand{\colFwidth}{0.17\textwidth}
\newcommand{\colGwidth}{0.5\textwidth}
\newcommand{\colHwidth}{0.28\textwidth}

% Used so that cross-references have a meaningful prefix
\newcounter{defnum} %Definition Number
\newcommand{\dthedefnum}{GD\thedefnum}
\newcommand{\dref}[1]{GD\ref{#1}}
\newcounter{datadefnum} %Datadefinition Number
\newcommand{\ddthedatadefnum}{DD\thedatadefnum}
\newcommand{\ddref}[1]{DD\ref{#1}}
\newcounter{theorynum} %Theory Number
\newcommand{\tthetheorynum}{T\thetheorynum}
\newcommand{\tref}[1]{T\ref{#1}}
\newcounter{tablenum} %Table Number
\newcommand{\tbthetablenum}{T\thetablenum}
\newcommand{\tbref}[1]{TB\ref{#1}}
\newcounter{assumpnum} %Assumption Number
\newcommand{\atheassumpnum}{P\theassumpnum}
\newcommand{\aref}[1]{A\ref{#1}}
\newcounter{goalnum} %Goal Number
\newcommand{\gthegoalnum}{P\thegoalnum}
\newcommand{\gsref}[1]{GS\ref{#1}}
\newcounter{instnum} %Instance Number
\newcommand{\itheinstnum}{IM\theinstnum}
\newcommand{\iref}[1]{IM\ref{#1}}
\newcounter{reqnum} %Requirement Number
\newcommand{\rthereqnum}{P\thereqnum}
\newcommand{\rref}[1]{R\ref{#1}}
\newcounter{nfrnum} %NFR Number
\newcommand{\rthenfrnum}{NFR\thenfrnum}
\newcommand{\nfrref}[1]{NFR\ref{#1}}
\newcounter{lcnum} %Likely change number
\newcommand{\lthelcnum}{LC\thelcnum}
\newcommand{\lcref}[1]{LC\ref{#1}}
\newcounter{ulcnum} %Unlikely change number
\newcommand{\ltheulcnum}{ULC\theulcnum}
\newcommand{\ulcref}[1]{ULC\ref{#1}}

\usepackage{fullpage}

\newcommand{\deftheory}[9][Not Applicable]
{
\newpage
\noindent \rule{\textwidth}{0.5mm}

\paragraph{RefName: } \textbf{#2} \phantomsection 
\label{#2}

\paragraph{Label:} #3

\noindent \rule{\textwidth}{0.5mm}

\paragraph{Equation:}

#4

\paragraph{Description:}

#5

\paragraph{Notes:}

#6

\paragraph{Source:}

#7

\paragraph{Ref.\ By:}

#8

\paragraph{Preconditions for \hyperref[#2]{#2}:}
\label{#2_precond}

#9

\paragraph{Derivation for \hyperref[#2]{#2}:}
\label{#2_deriv}

#1

\noindent \rule{\textwidth}{0.5mm}

}

\begin{document}

\title{Software Requirements Specification for \\ \progname: Save money on every trip} 
\author{\authname}
\date{\today}
	
\maketitle

~\newpage

\pagenumbering{roman}

\tableofcontents

~\newpage

\section*{Revision History}

\begin{tabularx}{\textwidth}{p{3cm}p{2cm}X}
\toprule {\bf Date} & {\bf Version} & {\bf Notes}\\
\midrule
05-10-2022 & 1.0 & Software Requirements Specification Version 1\\
\bottomrule
\end{tabularx}

~\newpage

\pagenumbering{arabic}

\section{Introduction}

Navigation applications are commonly used applications while driving to get
directions from point A to B, but these applications never tell you how much it
costs you to get there or how much gas was used on the trip. When carpooling
with friends, the driver of the vehicle always asks everyone in the group for gas
money and often times these calculations are mere estimates that are not always
very accurate. People often wonder how much it costs to get to a destination
before starting your journey, having an application perform these cost and fuel
calculations based on real time gas pricing information ensures you save the
most money on your journey while minimizing gas usage to encourage a more
sustainable lifestyle. The introduction section focuses on the" Purpose of the Document", 
the "Scope of Requirements", the "Characteristics of the Intended Reader", 
and the "Organization of Document". 

\subsection{Purpose of Document}

The purpose of this document is to outline the overall description and target functionality
of the application to stakeholders. It will communicate, without any ambiguity, the system
requirements for Greenway and the different use cases and contexts of use. These design
aspects will be communicated using a combination of diagrams and written descriptions.
In addition, it will also serve as a reference for verifying correctness and validation of the
product during the testing phases.


\subsection{Characteristics of Intended Reader} \label{sec_IntendedReader}

The intended readers characteristics include having knowledge in the following domains, 
the understanding of mapping software, the knowledge of how data is stored and will be used
in the context of the google maps API, which algorithms need to be used to calculate the most
fuel efficient route, the understanding of how terrain information will be used to alter car mileage
calculations, and the understanding of how information will be retrieved for all types of data being used.
The intended reader needs this knowledge as it is required to understand the rest of the SRS document;
this ensures the reader understands what the information is that is being conveyed and how the technology
is going to be used to put out the final product.

\subsection{Organization of Document}

The document is organized into sections with smaller subsection that will aid in conveying information.
The intention is that these sections will work together to provide a more informed understanding of the product.
The document is seperated into sections that include, "General System Description", "Specific System Description", "Requirements",
"Likely Changes", "Unlikely Changes", "Traceability Matrices and Graphs", "Development Plan", and the "Values of Auxiliary Constants".

\section{Project Drivers}

\subsection{The Purpose of the Project}

The goal of Greenway is to create an application that allows users to provide car information and a destination, in return the application supplies the amount of money it will take for the user to reach that destination using the most fuel-efficient route calculated by the app. The application uses real-time gas price data, information on mileage data, and terrain data to calculate and provide the amount of money it would take to reach a destination for users who drive many different types of gas vehicles; this allows them to use this app in many ways, such as, figuring out which gym may provide the most bang for their buck based on distance from their location.

\subsection{Scope of Requirements} 

The scope of the proposed software, Greenway, is a mapping software that not only 
gives fuel efficient directions to the intended destination but provides the user with fuel cost calculations.
The intention is to calculate fuel costs using gas price data, car mileage information and terrain information
to show how much money it will take to get to the intended destination using the most fuel efficient route.

\subsection{Stakeholders}

\subsubsection{Product Users}
The product user is anyone who can drive and hence will be using the product. The product users have the most influence on the requirements of the application and its overall development since the purpose of the application is help these users choose an cost effective eco-friendly route for a journey.

\subsubsection{Development Team}
The development team is the group of people who are developing the application. They are the software developers who are developing the application and hence are an important stakeholder.

\subsubsection{Teaching Staff/Instructors}
The teaching staff and instructors are also important stake holders as they will be directly interacting with our product and assessing its usability. They are similar to investors and are required to be shown updates and hold influence on the requirements of the application and its development.

\section{General System Description}

This section provides general information about the system.  It identifies the
interfaces between the system and its environment and describes the user
characteristics.

\subsection{System Context}

\begin{figure}[h!]
\begin{center}
 \includegraphics[width=0.8\textwidth]{ContextDiagram.png}
\caption{Context Diagram}
\label{Fig_SystemContext} 
\end{center}
\end{figure}


\begin{itemize}
\item User Responsibilities:
\begin{itemize}
\item Responsible for interacting with Application Frontend and receiving its output. 
\end{itemize}
\item \progname{} System Responsibilities:
\begin{itemize}
\item Responsible for calculating the total fuel price for trip with data from 
Feul Data API.
\item Responsible for interacting with GPS API to get all neccessary GPS data 
to present to user.
\item Responsible for interacting with Car Data API to get car information 
and process it.
\item Responsible for interacting with Charging station API to get charging 
station data.
\end{itemize}
\item Application Frontend Responsibilities:
\begin{itemize}
\item Responsible for allowing user to interact with application and displaying application output.
\end{itemize}
\item Fuel Data API Responsibilities:
\begin{itemize}
\item Responsible for sending Feul Station Data to any service requesting it.
\end{itemize}
\item GPS System Responsibilities:
\begin{itemize}
\item Responsible for sending any GPS data required to a service needing it.
\end{itemize}
\item Car Data API Responsibilities:
\begin{itemize}
\item Responsible for sending any Car data required to a service needing it.
\end{itemize}
\item Charging API Responsibilities:
\begin{itemize}
\item Responsible for sending any Charging Station data required to a service needing it.
\end{itemize}
\end{itemize}

\subsection{User Characteristics} \label{SecUserCharacteristics}

The users of this application are individuals who own gas powered vehicles and are looking for information on how much it will cost them in terms of fuel. This application will serve as a way of calculating the most fuel-efficient route taking into consideration elevation data, fuel mileage from an individuals car model, and the current gas prices. This users of app will have the ability to put in their car’s information, giving them all the data they require to make an informed decision on if the destination they chose is worth going to.

\newpage
\section{Behavior}

\begin{figure}[h!]
  \begin{center}
   \includegraphics[width=0.8\textwidth]{DataFlowDiagram.png}
  \caption{Application Behavior from a Data Flow Diagram}
  \label{Fig_DataFlow} 
  \end{center}
\end{figure}

The Data flow diagram is used to demonstrate how the app will behave when data is provided to it through the user requesting a route. More specifically the frontend will access the main process with the request and then it will calculate everything it needs to with multiple different processes it accesses. Namely the Fuel Process will give it data regarding the fuel prices, GPS will give all positioning data and elevation along with anything else the program displays, and car data will provide car efficiency data. All of which will be used to calculate the final route and price to display on the front end. 

\newpage
\section{Requirements}

This section provides the functional requirements, the business tasks that the
software is expected to complete, and the nonfunctional requirements, the
qualities that the software is expected to exhibit.

\subsection{Functional Requirements}

\noindent \begin{itemize}

\item[FR\refstepcounter{reqnum}\thereqnum. ] The system shall allow the user a way to input a start location.
\begin{itemize}
    \item Rationale: The application must know where the trip begins.
    \item Constants: None
    \item Undesired Event Handling: Display an error message and stop application.
    \item Normal Operation: The system is considered to be operating normally if the user can input a starting location for the trip.
	\item Priority: High
\end{itemize}

\item[FR\refstepcounter{reqnum}\thereqnum. ] The system shall allow the user a way to input a final destination.
\begin{itemize}
    \item Rationale: The application must know where the trip ends.
    \item Constants: None
    \item Undesired Event Handling: Display an error message and stop application.
    \item Normal Operation: The system is considered to be operating normally if the user can input a end location for the trip.
	\item Priority: High
\end{itemize}

\item[FR\refstepcounter{reqnum}\thereqnum. ] The system shall allow the user a way to select any car model and make.
\begin{itemize}
    \item Rationale: The system can use the user’s car model to internally calculate the fuel economy of the vehicle they will be driving in.
    \item Constants: None
    \item Undesired Event Handling: Assume the vehicle has the national average fuel economy of 8.9 L/100km.
    \item Normal Operation: The system is considered to be operating normally if the user can select any car by make/model and the system uses the corresponding fuel economy for its calculations.
	\item Priority: High
\end{itemize}

\item[FR\refstepcounter{reqnum}\thereqnum. ] The system shall allow the user a way to input the fuel economy of their car.
\begin{itemize}
    \item Rationale: The system can use a custom user-entered value to calculate the fuel economy.
    \item Constants: None
    \item Undesired Event Handling: Assume the vehicle has the national average fuel economy of 8.9 L/100km.
    \item Normal Operation: The system is considered to be operating normally if the user can input a numeric value for the fuel economy.
	\item Priority: High
\end{itemize}

\item[FR\refstepcounter{reqnum}\thereqnum. ] The system shall have a way to update available car models from either user input or external sources.
\begin{itemize}
    \item Rationale: The models available every year change and the system needs to account for newer cars entering the market.
    \item Constants: None
    \item Undesired Event Handling: If updates are unsuccessful, continue using the existing database of cars.
    \item Normal Operation: The system is considered to be operating normally if the user can update the fuel economy of a car.
	\item Priority: Low
\end{itemize}

\item[FR\refstepcounter{reqnum}\thereqnum. ] The system shall have a map to show journey from start to final destination.
\begin{itemize}
    \item Rationale: The user would appreciate having a visual of the route on a map.
    \item Constants: None
    \item Undesired Event Handling: Display a message stating “Unable to display map”.
    \item Normal Operation: The system is considered to be operating normally if the user can view a map of the overview of the route.
	\item Priority: Medium
\end{itemize}

\item[FR\refstepcounter{reqnum}\thereqnum. ] The system shall display the gas stations along the route.
\begin{itemize}
    \item Rationale: The user can easily see where to refuel if necessary.
    \item Constants: None
    \item Undesired Event Handling: Display an error message “Unable to locate Gas Stations.”
    \item Normal Operation: The system is considered to be operating normally if the map displays gas stations at their locations, indicated with a universal gas icon.
	\item Priority: Low
\end{itemize}

\item[FR\refstepcounter{reqnum}\thereqnum. ] The system shall display charging stations along the route for electric vehicles.
\begin{itemize}
    \item Rationale: This accommodates users with electric vehicles so they can make use of the system.
    \item Constants: None
    \item Undesired Event Handling: Display an error “Unable to locate charging stations.”
    \item Normal Operation: The system is considered to be operating normally if the map displays charging stations at their locations, indicated with a universal electricity icon.
	\item Priority: Low
\end{itemize}

\item[FR\refstepcounter{reqnum}\thereqnum. ] The system shall display gas prices at stations on route to destination.
\begin{itemize}
    \item Rationale: A key feature of the system is to have real-time gas prices factor into the cost calculations.
    \item Constants: None
    \item Undesired Event Handling: Use most recent price available if real-time value is not available.
    \item Normal Operation: The system is considered to be operating normally if the map displays gas stations at their locations, along with the price listed at each station.
	\item Priority: High
\end{itemize}

\item[FR\refstepcounter{reqnum}\thereqnum. ] The system shall have a way to calculate the most fuel efficient route.
\begin{itemize}
    \item Rationale: The target user would like to minimize their fuel costs.
    \item Constants: None
    \item Undesired Event Handling: Display an error message “Cannot calculate route.”
    \item Normal Operation: The system is considered to be operating normally if the map displays an optimized route from the defined start and end points.
	\item Priority: High
\end{itemize}

\item[FR\refstepcounter{reqnum}\thereqnum. ] The system shall have a way to collect elevation data along the suggested route.
\begin{itemize}
    \item Rationale: The system must use elevation data to accurately determine fuel costs based on more uphill/downhill driving required.
    \item Constants: None
    \item Undesired Event Handling: Ignore the elevation if unavailable and assume flat drive.
    \item Normal Operation: The system is considered to be operating normally if different points on the map have appropriate elevations.
	\item Priority: High
\end{itemize}

\item[FR\refstepcounter{reqnum}\thereqnum. ] The system shall have a way to calculate fuel consumption on different terrain elevations.
\begin{itemize}
    \item Rationale: The system must use elevation data to accurately determine fuel costs based on more uphill/downhill driving required.
    \item Constants: None
    \item Undesired Event Handling: Ignore the elevation if unavailable and assume flat drive.
    \item Normal Operation: The system is considered to be operating normally if the fuel costs increase with more uphill routes and decrease with more downhill routes.
	\item Priority: High
\end{itemize}

\item[FR\refstepcounter{reqnum}\thereqnum. ] The system must calculate the total cost of the journey.
\begin{itemize}
    \item Rationale: The user must know how much their trip costs.
    \item Constants: None
    \item Undesired Event Handling: Display a message “An error was encountered calculating the cost”
    \item Normal Operation: The system is considered to be operating normally if there is an output displaying the total cost of the trip after performing the appropriate calculations.
	\item Priority: High
\end{itemize}

\end{itemize}

\subsection{Non-functional Requirements}

\noindent \begin{itemize}

\item[NFR\refstepcounter{nfrnum}\thenfrnum \label{NFR_1}:] 
\begin{itemize}
  \item \textit{Description}: The system shall have a modern minimalist user interface.
  \item \textit{Rationale}: Using popular modern design principles to create an appealing visual style for the system will make the system easy to use and attractive to users.
  \item \textit{Fit Criterion}: The system’s UI must follow modern minimalist design principles as defined in popular style guides. Users feedback will be used to assess the overall design of the interface. 
  \item \textit{Undesired Event Handling}: Given poor user feedback on interface design, user suggestions will be considered and overall interface design will be reevaluated. 
  \item \textit{Priority}: Medium
\end{itemize}


\item[NFR\refstepcounter{nfrnum}\thenfrnum \label{NFR_2}:] 
\begin{itemize}
  \item \textit{Description}: The system shall have a consistent style across the different pages (or interfaces) of the platform. 
  \item \textit{Rationale}: System will appear more professional. User experience will be more streamlined and less confusing if the system maintains a cohesive style. 
  \item \textit{Fit Criterion}: Each page/interface shall be checked against a list of common style/design elements that should remain consistent within the project. User feedback will be used to assess style consistency between interfaces. 
  \item \textit{Undesired Event Handling}: Given feedback on inconsistent style, alterations will be made to the interface to address specific feedback. 
  \item \textit{Priority}: Medium
\end{itemize}

\item[NFR\refstepcounter{nfrnum}\thenfrnum \label{NFR_3}:] 
\begin{itemize}
  \item \textit{Description}: The system shall have a user interface which is intuitive and easy to navigate. 
  \item \textit{Rationale}: Ease of use improves learnability and memorability of the system.
  \item \textit{Fit Criterion}: User feedback will be used to assess ease of use. Metrics such as time taken for new user to complete sample task will be measured.
  \item \textit{Undesired Event Handling}: Given poor ease of use metrics, the layout, simplicity, and flow of system will be reevaluated and alterations made. 
  \item \textit{Priority}: High
\end{itemize}

\item[NFR\refstepcounter{nfrnum}\thenfrnum \label{NFR_4}:] 
\begin{itemize}
  \item \textit{Description}: The system shall retain the user’s previous destinations. 
  \item \textit{Rationale}: Reduces time taken to select destination if user has frequently used destinations. Improves ease of use (NFR 3) by reducing time needed to complete task. 
  \item \textit{Fit Criterion}: Are the user’s previous destinations available to the user? Yes/No.  
  \item \textit{Undesired Event Handling}: If no, functionality should be implemented.
  \item \textit{Priority}: Medium
\end{itemize}

\item[NFR\refstepcounter{nfrnum}\thenfrnum \label{NFR_5}:] 
\begin{itemize}
  \item \textit{Description}: The system shall have a guide (or short tutorial) for those who are using the application for the first time. 
  \item \textit{Rationale}: The guide/tutorial will introduce how to use all the features offered by the system.
  \item \textit{Fit Criterion}: Several metrics measuring task completion will be compared: first time user without tutorial, first time user with tutorial, repeat user without tutorial. User feedback will also be obtained. Majority of users \(\>80\%\) should feel comfortable with system functionality after viewing the tutorial.  
  \item \textit{Undesired Event Handling}: Given poor learnability, tutorial/guide and overall layout and ease of use (NFR 3) shall be reevaluated and alterations made.
  \item \textit{Priority}: High
\end{itemize}

\item[NFR\refstepcounter{nfrnum}\thenfrnum \label{NFR_6}:] 
\begin{itemize}
  \item \textit{Description}: The system shall hide the details of its implementation from the user.
  \item \textit{Rationale}: The implementation (e.g. code, algorithms, etc.), is only relevant to the developers. Users are only concerned about the application functionality and outputs.
  \item \textit{Fit Criterion}: During software testing, developers can test to see if abstraction was properly used to hide the details of the implementation of the system from the user.  
  \item \textit{Undesired Event Handling}: Given poor abstraction, details will be hidden or omitted from the user interface. 
  \item \textit{Priority}: High
\end{itemize}

\item[NFR\refstepcounter{nfrnum}\thenfrnum \label{NFR_7}:] 
\begin{itemize}
  \item \textit{Description}: The system shall be accessible to users with visual impairments. 
  \item \textit{Rationale}: The system shall be accessible to users with visual impairments. 
  \item \textit{Fit Criterion}: The system will be tested in conditions where the user is partially or completely visually impaired. 
  \item \textit{Undesired Event Handling}: Given poor accessibility, modifications to the system will be made to improve the usability of the system under these conditions.
  \item \textit{Priority}: Medium
\end{itemize}

\item[NFR\refstepcounter{nfrnum}\thenfrnum \label{NFR_8}:] 
\begin{itemize}
  \item \textit{Description}: The system shall complete the route and fuel consumption calculation in a reasonable amount of time (<5 seconds). 
  \item \textit{Rationale}: The user should not have to wait a very long time after they input their destination into the application.
  \item \textit{Fit Criterion}: The time between the user input and the route calculation shall not exceed 5 seconds. The system calculation time should be comparable to the default google maps route calculation time.  
  \item \textit{Undesired Event Handling}: Given poor calculation time, algorithm time efficiency and other sources of lag should be considered and fixed (if possible). 
  \item \textit{Priority}: Low
\end{itemize}

\item[NFR\refstepcounter{nfrnum}\thenfrnum \label{NFR_9}:] 
\begin{itemize}
  \item \textit{Description}: The system shall calculate fuel consumption as accurately as possible.
  \item \textit{Rationale}: We want our customers to know exactly how much fuel will be consumed during the trip. It is vital that the value be accurate so that they can plan their trip accordingly.
  \item \textit{Fit Criterion}: In a panel of 50 users using the application, 80\% of the users are satisfied with the calculation.
  \item \textit{Undesired Event Handling}: N/A.
  \item \textit{Priority}: Medium
\end{itemize}

\item[NFR\refstepcounter{nfrnum}\thenfrnum \label{NFR_10}:] 
\begin{itemize}
  \item \textit{Description}: The system shall be available for use 24 hours per day, 365 days per year.
  \item \textit{Rationale}: Since this is a web app, it will be available for use at any time. Users will be able to use the application at any time of the day without any restriction.
  \item \textit{Fit Criterion}: 90\% of a panel of 10 users shall be able to use the application without it going offline.
  \item \textit{Undesired Event Handling}: N/A.
  \item \textit{Priority}: Low
\end{itemize}

\item[NFR\refstepcounter{nfrnum}\thenfrnum \label{NFR_11}:] 
\begin{itemize}
  \item \textit{Description}: The system shall be able to respond to invalid or unconventional inputs.  
  \item \textit{Rationale}: Ensures user error will not cause bugs or impede system functionality. 
  \item \textit{Fit Criterion}: Test cases with invalid inputs and outliers will be created to test this non-functional requirement.  
  \item \textit{Undesired Event Handling}: N/A. 
  \item \textit{Priority}: High
\end{itemize}

\item[NFR\refstepcounter{nfrnum}\thenfrnum \label{NFR_12}:] 
\begin{itemize}
  \item \textit{Description}: The system shall store the last 10 addresses that the user input and display it in the search bar.
  \item \textit{Rationale}: For the user’s convenience, the last 10 search results shall be displayed in case the user wants to go to any of those locations again.
  \item \textit{Fit Criterion}: After a new address is searched, that search result is added in the last 10 recent results. After 10 searches with different addresses, the oldest search result is overwritten.
  \item \textit{Undesired Event Handling}: In case nothing shows up, the system shall just show a regular search bar with no history.
  \item \textit{Priority}: Low
\end{itemize}

\item[NFR\refstepcounter{nfrnum}\thenfrnum \label{NFR_13}:] 
\begin{itemize}
  \item \textit{Description}: The system shall be able to handle an infinite amount of users.
  \item \textit{Rationale}: This system shall stay available for anyone who is concerned about spending too much for a particular trip.
  \item \textit{Fit Criterion}: As a local server will be hosting the web app, all users should be able to access the application without being blocked.
  \item \textit{Undesired Event Handling}: In case nothing shows up, the system shall just show a regular search bar with no history.
  \item \textit{Priority}: Medium
\end{itemize}

\item[NFR\refstepcounter{nfrnum}\thenfrnum \label{NFR_14}:] 
\begin{itemize}
  \item \textit{Description}: The system shall operate as long as possible.
  \item \textit{Rationale}: This system shall stay up indefinitely for users who are concerned about how much gas will be used up during their trip.
  \item \textit{Fit Criterion}: 9 out of 10 users shall be able to use the application without it being unusable.
  \item \textit{Undesired Event Handling}: N/A
  \item \textit{Priority}: Low
\end{itemize}

\item[NFR\refstepcounter{nfrnum}\thenfrnum \label{NFR_15}:] 
\begin{itemize}
  \item \textit{Description}: The system shall be able to run on all modern browsers. 
  \item \textit{Rationale}: Users with the system on many different platforms will be able to use it. 
  \item \textit{Fit Criterion}: Opening the HTML executable file with all different modern browsers will work.
  \item \textit{Undesired Event Handling}: N/A
  \item \textit{Priority}: Medium
\end{itemize}

\item[NFR\refstepcounter{nfrnum}\thenfrnum \label{NFR_16}:] 
\begin{itemize}
  \item \textit{Description}: The system shall have a simple installation process for all potential users.
  \item \textit{Rationale}: This system shall have a simple installation guide provided with easy steps so that any user should be able to install the web app.
  \item \textit{Fit Criterion}: 90 percent of a panel of 20 potentials users will be able to install the application without any hassle.
  \item \textit{Undesired Event Handling}: In case of installation errors, they can refer to the documentation on GitHub about alternative ways to install.
  \item \textit{Priority}: High
\end{itemize}

\item[NFR\refstepcounter{nfrnum}\thenfrnum \label{NFR_17}:] 
\begin{itemize}
  \item \textit{Description}: The system shall be available to download for the public via GitHub.
  \item \textit{Rationale}: This system shall be distributed from the GitHub website for free for everyone.
  \item \textit{Fit Criterion}: All users should be available to pull the repository and install the application in their personal device.
  \item \textit{Undesired Event Handling}: N/A
  \item \textit{Priority}: Medium
\end{itemize}

\item[NFR\refstepcounter{nfrnum}\thenfrnum \label{NFR_18}:] 
\begin{itemize}
  \item \textit{Description}: The system’s code shall be well documented and clear to be easily understandable for any developer maintaining the code.
  \item \textit{Rationale}: If a new developer went to work on the piece of the system, they should be able to easily understand what was previously worked on.
  \item \textit{Fit Criterion}: When inspecting the code, comments will be found describing what that section does.
  \item \textit{Undesired Event Handling}: If any documentation is missing, the developer of the code shall be contacted as soon as possible to provide the corresponding documentation.
  \item \textit{Priority}: High
\end{itemize}

\item[NFR\refstepcounter{nfrnum}\thenfrnum \label{NFR_19}:] 
\begin{itemize}
  \item \textit{Description}: Supporting documentation shall be available on GitHub, which includes a user’s guide on how to use the application.
  \item \textit{Rationale}: In order for users to understand how to download, launch the application, and access relevant data, there should be documentation to explain.
  \item \textit{Fit Criterion}: When downloading the system, the supporting documentation will be found on the GitHub site. 
  \item \textit{Undesired Event Handling}: If any documentation is missing, the developer of the code shall be contacted as soon as possible to provide the corresponding documentation.
  \item \textit{Priority}: Medium
\end{itemize}

\item[NFR\refstepcounter{nfrnum}\thenfrnum \label{NFR_20}:] 
\begin{itemize}
  \item \textit{Description}: The system shall be compatible to run between Windows, Mac, and Linux-based operating systems.
  \item \textit{Rationale}: In order to support a wide variety of users, we need to allow users on the most commonly used operating systems to download and use the system.
  \item \textit{Fit Criterion}: Downloading and running the files on all mentioned operating systems will show compatibility.
  \item \textit{Undesired Event Handling}: If the application does not work on a certain operating system, a support tool will be provided to contact the developers to fix the issue.
  \item \textit{Priority}: Medium
\end{itemize}

\item[NFR\refstepcounter{nfrnum}\thenfrnum \label{NFR_21}:] 
\begin{itemize}
  \item \textit{Description}: System should comply with the Google Maps Platform Terms of Service.
  \item \textit{Rationale}: Mandatory compliance requirement when using the Google Maps Platform API. 
  \item \textit{Fit Criterion}: The system will be checked directly against each point in the terms of service.
  \item \textit{Undesired Event Handling}: If the system, at any point, does not comply with the terms of service, the system will be changed to comply with the terms.
  \item \textit{Priority}: High
\end{itemize}

\end{itemize}

\newpage

\section{Likely Changes}    

\noindent \begin{itemize}

\item[LC\refstepcounter{lcnum}\thelcnum\label{Long Journey}:] The addition of the application being able to use mileage data as well as gas tank size to calculate the most efficient route on a long journey, allowing for the least amount of stops to fill up gas.

\item[LC\refstepcounter{lcnum}\thelcnum\label{Electric Cars}:] The addition of the application taking data of electric cars to allow the use of not only gas station but charging stations for electric car users.

\item[LC\refstepcounter{lcnum}\thelcnum\label{Bigger Area}:] The application will be able to be used in all of Ontario with elevation and gas data for all areas.

\item[LC\refstepcounter{lcnum}\thelcnum\label{Application Integration}:] Future changes should allow the application to be used with other apps other than Google Maps, such as Wayz or Apple Maps.

\end{itemize}

\section{Unlikely Changes}    

\noindent \begin{itemize}

\item[ULC\refstepcounter{ulcnum}\theulcnum\label{Regional Support}:] The application will not support other provinces or countries.

\item[ULC\refstepcounter{ulcnum}\theulcnum\label{New Cars}:] The application will not be able to update car catalog information on its own and will require an update to the application when new cars are introduced.

\item[ULC\refstepcounter{ulcnum}\theulcnum\label{Hybrid Cars}:] The application will not support hybrid vehicles as they require complex calculations to determine when the electric component of the vehicle is used and when the gas component is.

\end{itemize}

\newpage

\section{Traceability Matrices and Graphs}

The purpose of the traceability matrices is to provide easy references on what
has to be additionally modified if a certain component is changed.  Every time a
component is changed, the items in the column of that component that are marked
with an ``X'' may have to be modified as well.  Table~\ref{Table:trace} shows the
dependencies of Functional Requirements and Non-Functional Requirements.

\begin{table}[h!]
\centerline{%
\begin{tabular}{l|ccccccccccccc}
\multicolumn{1}{c|}{NFR IDs} & \multicolumn{12}{c}{FR IDs} \\ \hline
 & \multicolumn{1}{l}{FR1} & \multicolumn{1}{l}{FR2} & \multicolumn{1}{l}{FR3} & \multicolumn{1}{l}{FR4} & \multicolumn{1}{l}{FR5} & \multicolumn{1}{l}{FR6} & \multicolumn{1}{l}{FR7} & \multicolumn{1}{l}{FR8} & \multicolumn{1}{l}{FR9} & \multicolumn{1}{l}{FR10} & \multicolumn{1}{l}{FR11} & \multicolumn{1}{l}{FR12} & \multicolumn{1}{l}{FR13} \\ \hline
NFR1  & X & X & X & X &  & X & X & X & X &  &  &  & \\ \hline
NFR2  & X & X & X & X &  & X & X & X & X &  &  &  & \\ \hline
NFR3  & X & X & X & X &  & X & X & X & X &  &  &  & \\ \hline
NFR4  &  &  &  &  &  & X &  &  &  &  &  &  &  \\ \hline
NFR5  & X & X & X & X &  & X & X & X & X &  &  &  & \\ \hline
NFR6  &  &  &  &  & X &  &  &  &  & X & X & X & X \\ \hline
NFR7  & X & X & X & X &  & X & X & X & X &  &  &  & \\ \hline
NFR8  &  &  &  &  &  &  &  &  &  & X & X & X & X  \\ \hline
NFR9  &  &  &  & X &  &  &  &  &  &  &  &  & X \\ \hline
NFR10 &  &  &  &  &  &  &  &  &  &  &  &  &  \\ \hline
NFR11 & X & X & X & X &  &  &  &  &  &  &  &  &  \\ \hline
NFR12 &  &  &  &  &  & X &  &  &  &  &  &  &  \\ \hline
NFR13 &  &  &  &  &  &  &  &  &  &  &  &  &  \\ \hline
NFR14 &  &  &  &  &  &  &  &  &  &  &  &  &  \\ \hline
NFR15 &  &  &  &  &  &  &  &  &  &  &  &  &  \\ \hline
NFR16 &  &  &  &  &  &  &  &  &  &  &  &  &  \\ \hline
NFR17 &  &  &  &  &  &  &  &  &  &  &  &  &  \\ \hline
NFR18 &  &  &  &  &  &  &  &  &  &  &  &  &  \\ \hline
NFR19 &  &  &  &  &  &  &  &  &  &  &  &  &  \\ \hline
NFR20 &  &  &  &  &  &  &  &  &  &  &  &  &  \\ \hline
NFR21 &  &  &  &  &  &  &  &  &  &  &  &  &  \\
\end{tabular}
}
\caption{Traceability Matrix: Functional Requirements to Non-Functional Requirements}
\label{Table:trace}
\end{table}

\newpage

\bibliographystyle {plainnat}
\bibliography {../../refs/References}

\newpage

\section*{Appendix --- Reflection}

There are many skills needed to accomplish the task, these skills however need to be distributed amongst the team as it makes it easier to complete the goals that have been set to acquire a finished product. The domain specific knowledge required ranges from understanding APIs to collecting, storing, and using data. The team requires the knowledge of how to collect the different types of data required, gas price data, elevation data of a specified area, car catalog data that contains information about gas mileage, and google maps data for fuel efficient route calculations. The intention is to use this data as a collective to perform calculations which would allow the user to find the cost of travel from one destination to another in the simplest manner; for this, an understanding of how the Google Maps API works, how the application will be developed, how the interface will be designed, and how everything will work simultaneously to provide the user with the best experience. These tasks are split amongst the team.

\bigskip

\noindent Jash --- Collection of gas price data and how it will be stored and how the fuel calculation will be performed.

\bigskip

\noindent Priyansh --- Collection of elevation specific data and how it will be stored and how the fuel calculation will be performed.

\bigskip

\noindent Utsharga --- Collection of car catalog data and how it will be stored and how the fuel calculation will be performed.

\bigskip

\noindent Sharjil --- Understanding of how the interface will be designed and how the fuel-efficient route will be calculated.

\bigskip

\noindent Bilal --- Understanding of how the Google Maps API works and how the fuel-efficient route will be calculated.

\bigskip

\noindent Pranay --- Understanding How the interface will be designed and how everything will work simultaneously.

\bigskip

There are many approaches to acquiring this knowledge; one approach is researching in the specific domain through forums and google searches. Another approach is getting hands on with the entity you would like to learn, this allows one to experience and interact with it, giving a greater understanding of how everything works.

\bigskip

\noindent Jash --- The approach for collecting gas price data will be through the research domain, this allows him to find information about the gas prices as well as learn how to collect and store them in a specific way. The same approach applies to the fuel calculations as understanding how its done and factoring in the elevation data is going to be used.

\bigskip

\noindent Priyansh --- The approach for collecting elevation data will be done through the research domain, this makes sure the data being collected is relevant to the information required to perform the fuel calculations. The same approach applies to the fuel calculations as understanding how its done and factoring in the elevation data is going to be used.

\bigskip

\noindent Utsharga --- The approach for collecting car catalog data will be done through the research domain, this allows him to collect all necessary data on all existing cars to be used in production. The same approach applies to the fuel calculations as understanding how its done and factoring in the elevation data is going to be used.

\bigskip

\noindent Sharjil --- The design of the user interface will be done through the hand-on domain, this allows him to interact with the product and test to see what works and what doesn’t. The calculations for the fuel-efficient route will be done through the research domain, this allows him to create and work on the algorithm required to perform the calculation.

\bigskip

\noindent Bilal --- The Google Maps API will require the hand-on domain to be used to explore how it works and how it will be used for the purposes of this application, this will allow him to learn what to do and how to use it for our needs. The calculations for the fuel-efficient route will be done through the research domain, this allows him to create and work on the algorithm required to perform the calculation.

\bigskip

\noindent Pranay --- The design of the user interface will be done through the hand-on domain, this allows him to interact with the product and test to see what works and what doesn’t. To get everything working together, he will have to look into the hand-on domain to be able to test and see how all the different parts will need to be integrated together to allow for the product to function as intended.


\end{document}