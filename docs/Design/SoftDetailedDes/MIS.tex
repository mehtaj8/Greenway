\documentclass[12pt, titlepage]{article}

\usepackage{amsmath, mathtools}

\usepackage[round]{natbib}
\usepackage{amsfonts}
\usepackage{amssymb}
\usepackage{graphicx}
\usepackage{colortbl}
\usepackage{xr}
\usepackage{hyperref}
\usepackage{longtable}
\usepackage{xfrac}
\usepackage{tabularx}
\usepackage{float}
\usepackage{siunitx}
\usepackage{booktabs}
\usepackage{multirow}
\usepackage[section]{placeins}
\usepackage{caption}
\usepackage{fullpage}

\hypersetup{
bookmarks=true,     % show bookmarks bar?
colorlinks=true,       % false: boxed links; true: colored links
linkcolor=red,          % color of internal links (change box color with linkbordercolor)
citecolor=blue,      % color of links to bibliography
filecolor=magenta,  % color of file links
urlcolor=cyan          % color of external links
}

\usepackage{array}

\externaldocument{../../SRS/SRS}

%% Comments

\usepackage{color}

\newif\ifcomments\commentstrue %displays comments
%\newif\ifcomments\commentsfalse %so that comments do not display

\ifcomments
\newcommand{\authornote}[3]{\textcolor{#1}{[#3 ---#2]}}
\newcommand{\todo}[1]{\textcolor{red}{[TODO: #1]}}
\else
\newcommand{\authornote}[3]{}
\newcommand{\todo}[1]{}
\fi

\newcommand{\wss}[1]{\authornote{blue}{SS}{#1}} 
\newcommand{\plt}[1]{\authornote{magenta}{TPLT}{#1}} %For explanation of the template
\newcommand{\an}[1]{\authornote{cyan}{Author}{#1}}

%% Common Parts

\newcommand{\progname}{ProgName} % PUT YOUR PROGRAM NAME HERE
\newcommand{\authname}{Team \#, Team Name
\\ Student 1 name and macid
\\ Student 2 name and macid
\\ Student 3 name and macid
\\ Student 4 name and macid} % AUTHOR NAMES                  

\usepackage{hyperref}
    \hypersetup{colorlinks=true, linkcolor=blue, citecolor=blue, filecolor=blue,
                urlcolor=blue, unicode=false}
    \urlstyle{same}
                                


\begin{document}

\title{Module Interface Specification for \progname{}}

\author{\authname{}}

\date{\today}

\maketitle

\pagenumbering{roman}

\section{Revision History}

\begin{tabularx}{\textwidth}{p{3cm}p{2cm}X}
\toprule {\bf Date} & {\bf Version} & {\bf Notes}\\
\midrule
Jan 18, 2023 & 1.0 & Revision 0\\
Date 2 & 1.1 & Notes\\
\bottomrule
\end{tabularx}

~\newpage

\section{Symbols, Abbreviations and Acronyms}

See SRS Documentation on \href{https://github.com/mehtaj8/Greenway/blob/main/docs/SRS/SRS.pdf}{project repo}.

\newpage

\tableofcontents

\newpage

\pagenumbering{arabic}

\section{Introduction}

The following document details the Module Interface Specifications for
\progname{}.

Complementary documents include the System Requirement Specifications
and Module Guide.  The full documentation and implementation can be
found at \url{https://github.com/mehtaj8/Greenway}.

\section{Notation}

The structure of the MIS for modules comes from HoffmanAndStrooper1995,
with the addition that template modules have been adapted from
GhezziEtAl2003.  The mathematical notation comes from Chapter 3 of
HoffmanAndStrooper1995.  For instance, the symbol := is used for a
multiple assignment statement and conditional rules follow the form $(c_1
\Rightarrow r_1 | c_2 \Rightarrow r_2 | ... | c_n \Rightarrow r_n )$.

The following table summarizes the primitive data types used by \progname. 

\begin{center}
\renewcommand{\arraystretch}{1.2}
\noindent 
\begin{tabular}{l l p{7.5cm}} 
\toprule 
\textbf{Data Type} & \textbf{Notation} & \textbf{Description}\\ 
\midrule
character & char & a single symbol or digit\\
integer & $\mathbb{Z}$ & a number without a fractional component in (-$\infty$, $\infty$) \\
natural number & $\mathbb{N}$ & a number without a fractional component in [1, $\infty$) \\
real & $\mathbb{R}$ & any number in (-$\infty$, $\infty$)\\
\bottomrule
\end{tabular} 
\end{center}

\noindent
The specification of \progname \ uses some derived data types: sequences, strings, and
tuples. Sequences are lists filled with elements of the same data type. Strings
are sequences of characters. Tuples contain a list of values, potentially of
different types. In addition, \progname \ uses functions, which
are defined by the data types of their inputs and outputs. Local functions are
described by giving their type signature followed by their specification.

\newpage

\section{Module Decomposition}

The following table is taken directly from the Module Guide document for this project.

\begin{table}[h]
\centering
\begin{tabular}{p{0.3\textwidth} p{0.6\textwidth}}
\toprule
\textbf{Level 1} & \textbf{Level 2}\\
\midrule

{Hardware-Hiding Module} & ~ \\
\midrule

\multirow{7}{0.3\textwidth}{Behaviour-Hiding Module} & Location Input Module\\
& Car Input Module\\
& Map Display Module\\
% & Fuel Information Module\\
& Trip Details Module\\
& Navigation Module\\
& SideBar Module\\
\midrule

\multirow{3}{0.3\textwidth}{Software Decision Module} & Map Data Module\\
& Car Data Module\\
& Fuel Data Module\\
\bottomrule

\end{tabular}
\caption{Module Hierarchy}
\label{TblMH}
\end{table}

\newpage

%%%%%%%%%%%%%%%%%%%%%%%%%%%%%%%%%%%%%%%%%%%%%%%%%%%%%%%%%%%%%%%%%%%%%%%%

\section{MIS of Location Input Module (M2)} 

\label{Module}

\subsection{Module extends React.Component}

Location\_Input\_Module

\subsection{Uses}

Map\_Data\_Module

\subsection{Syntax}

\subsubsection{Exported Constants}

None

\subsubsection{Inputs}
\begin{tabular}{| l | l | l |}
  \hline
  \textbf{Input Name} & \textbf{Type} & \textbf{Description}\\
  \hline
  Start & String & Address of the start location \\
  \hline
  Start & String & Address of the end location \\
  \hline
\end{tabular}

\subsubsection{Exported Access Programs}

\begin{tabular}{| l | l | l | l |}
  \hline
  \textbf{Name} & \textbf{In} & \textbf{Out} & \textbf{Exceptions}\\
  \hline
  constructor & HTML attribute & ~ & ~\\
  \hline
  setStartLocation & String Start & ~ & Location not found.\\
  \hline
  setEndLocation & String End & ~ & Location not found.\\
  \hline
  render & ~ & ReactDOM & ~\\
  \hline
\end{tabular}

\subsection{Semantics}

\subsubsection{State Variables}

$\mathit{props}: \text{HTML attribute}$\\

\subsubsection{Environment Variables}

None

\subsubsection{Assumptions}

None

\subsubsection{Access Routine Semantics}

\noindent constructor($props$):
\begin{itemize}
\item transition: $\mathit{state} = initialBlock$
\item output: $out := \mbox{self}$
\item exception: none
\end{itemize}

\noindent setStartLocation($String$ Start):
\begin{itemize}
\item transition: This function sets the start location from the user.
\item exception: Location not found.
\end{itemize}

\noindent setDestinationLocation($String$ End):
\begin{itemize}
\item transition: This function sets the destination location from the user.
\item exception: Location not found.
\end{itemize}

\noindent render():
\begin{itemize}
\item output: This function renders the output of the Input Location Module.
\item exception: none
\end{itemize}

\subsubsection{Local Functions}

None

\newpage

%%%%%%%%%%%%%%%%%%%%%%%%%%%%%%%%%%%%%%%%%%%%%%%%%%%%%%%%%%%%%%%%%%%%%%%%%%%%%%%%%%%%


%%%%%%%%%%%%%%%%%%%%%%%%%%%%%%%%%%%%%%%%%%%%%%%%%%%%%%%%%%%%%%%%%%%%%%%%

\section{MIS of Car Input Module (M3)} 

\label{Module} 

\subsection{Module extends React.Component}

Car\_Input\_Module

\subsection{Uses}

Car\_Data\_Module

\subsection{Syntax}

\subsubsection{Exported Constants}

None

\subsubsection{Inputs}
\begin{tabular}{| l | l | l |}
  \hline
  \textbf{Input Name} & \textbf{Type} & \textbf{Description}\\
  \hline
  Make & String & Make of the Car \\
  \hline
  Model & String & Model of the Car \\
  \hline
  Year & Number & Year of the Car \\
  \hline
  Mileage & String & Mileage of the Car \\
  \hline
\end{tabular}

\subsubsection{Exported Access Programs}

\begin{tabular}{| l | l | l | l |}
  \hline
  \textbf{Name} & \textbf{In} & \textbf{Out} & \textbf{Exceptions}\\
  \hline
  constructor & HTML attribute & ~ & ~\\
  \hline
  setCarMake & String & ~ & Make not found.\\
  \hline
  setCarModel & String & ~ & Model not found.\\
  \hline
  setCarYear & Number & ~ & Car not found.\\
  \hline
  setCarMileage & Number & ~ & Invalid mileage.\\
  \hline
  render & ~ & ReactDOM & ~\\
  \hline
\end{tabular}

\subsection{Semantics}

\subsubsection{State Variables}

$\mathit{props}: \text{HTML attribute}$\\

\subsubsection{Environment Variables}

None

\subsubsection{Assumptions}

None

\subsubsection{Access Routine Semantics}

\noindent constructor($props$):
\begin{itemize}
\item transition: $\mathit{state} = initialBlock$
\item output: $out := \mbox{self}$
\item exception: none
\end{itemize}

\noindent setCarMake($String$ Make):
\begin{itemize}
\item transition: This function sets the make of the car from the user by making a call to the database to find all available makes.
\item exception: Make not found.
\end{itemize}

\noindent setCarModel($String$ Model):
\begin{itemize}
\item transition: This function sets the model of the car from the user by making a call to the database to find all available Models.
\item exception: Model not found.
\end{itemize}

\noindent setCarYear($Number$ Year):
\begin{itemize}
\item transition: This function sets the manufacturing year of the car from the user by making a call to the database to find all available Years.
\item exception: Year not found.
\end{itemize}

\noindent setCarMileage($Number$ Mileage):
\begin{itemize}
\item transition: This function sets the mileage of the car from the user as an alternative the other car details.
\item exception: Year not found.
\end{itemize}

\noindent render():
\begin{itemize}
\item output: This function renders the output of the Input Car Module.
\item exception: none
\end{itemize}

\subsubsection{Local Functions}

None

\newpage

%%%%%%%%%%%%%%%%%%%%%%%%%%%%%%%%%%%%%%%%%%%%%%%%%%%%%%%%%%%%%%%%%%%%%%%%%%%%%%%%%%%%


%%%%%%%%%%%%%%%%%%%%%%%%%%%%%%%%%%%%%%%%%%%%%%%%%%%%%%%%%%%%%%%%%%%%%%%%

\section{MIS of Map Display Module (M4)} 

\subsection{Module}

Map\_Display\_Module, Location\_Input\_Module

\subsection{Uses}

Map\_Data\_Module

\subsection{Syntax}

\subsubsection{Exported Constants}

\subsubsection{Inputs}
\begin{tabular}{| l | l | l |}
  \hline
  \textbf{Input Name} & \textbf{Type} & \textbf{Description}\\
  \hline
  startEndLoc & List$<$String, String$>$ & Array of the start and end location as addresses that the user entered \\
  \hline
  mileageLocal & $\mathbb{R}$ & Price of gas locally \\
  \hline
  points & List$<\mathbb{R},\mathbb{R}>$  & Coordinates for points along the route \\
  \hline
  mileage & $\mathbb{R}$ & Mileage of the Car \\
  \hline
\end{tabular}

\subsubsection{Exported Access Programs}

\begin{tabular}{| l | l | l | l |}
  \hline
  \textbf{Name} & \textbf{In} & \textbf{Out} & \textbf{Exceptions}\\
  \hline
  getRoute & List$<$String, String$>$ startEndLoc & List$<\mathbb{R},\mathbb{R}>$  startEndLocCoords & Location not found.\\
  \hline
  calRoute & $\mathbb{R}$ mileageLocal & ~ & Make not found.\\
  \hline
  pathOverview & List$<\mathbb{R},\mathbb{R}>$ points, $\mathbb{R}$ mileage  & ReactDOM & Location not found.\\
  \hline
  getRoadGrade & List$<\mathbb{R},\mathbb{R}>$ Points & List$<\mathbb{R}>$ roadGrades & Make not found.\\
  \hline
  render & ~ & ReactDOM & ~\\
  \hline
\end{tabular}


\subsection{Semantics}

\subsubsection{State Variables}

Map\_state: $String$

\subsubsection{Environment Variables}

None

\subsubsection{Assumptions}

None

\subsubsection{Access Routine Semantics}

\noindent getRoute(List$<$String, String$>$ startEndLoc):
\begin{itemize}
\item output: List$<\mathbb{R},\mathbb{R}>$ : Coordinates of the start and destination locations.
\item exception: Location not found.
\end{itemize}

\noindent calRoute($String$ mileageLocal):
\begin{itemize}
\item transition: This function sets the model of the car from the user.
\item exception: Location not found.
\end{itemize}

\noindent pathOverview(List$<\mathbb{R},\mathbb{R}>$ points, $\mathbb{R}$ mileage):
\begin{itemize}
\item Output: The route from start to end destination.
\item exception: Map component not loaded.
\end{itemize}

\noindent getRoadGrade(List$<\mathbb{R},\mathbb{R}>$ points):
\begin{itemize}
\item Output: The road grade for each pair of coordiantes in points.
\item exception: Map component not loaded.
\end{itemize}

\noindent render():
\begin{itemize}
\item output: This function renders the output of the MAP component.
\item exception: none
\end{itemize}

\subsubsection{Local Functions}

None

\newpage

%%%%%%%%%%%%%%%%%%%%%%%%%%%%%%%%%%%%%%%%%%%%%%%%%%%%%%%%%%%%%%%%%%%%%%%%%%%%%%%%%%%%

%%%%%%%%%%%%%%%%%%%%%%%%%%%%%%%%%%%%%%%%%%%%%%%%%%%%%%%%%%%%%%%%%%%%%%%%

\section{MIS of Trip Detail Module (M5)} 

\label{Module} 

\subsection{Module extends React.Component}

TripDetails\_Module

\subsection{Uses}


\subsection{Syntax}

\subsubsection{Exported Constants}

\subsubsection{Inputs}
\begin{tabular}{| l | l | l |}
  \hline
  \textbf{Input Name} & \textbf{Type} & \textbf{Description}\\
  \hline
  GasPrice & $\mathbb{R}$ & Price of gas in the start location \\
  \hline
  mileage & $\mathbb{R}$ & Mileage of car \\
  \hline
  Distance & $\mathbb{R}$  & Total distance to travel along the route in km \\
  \hline
  Total Price & $\mathbb{R}$ & Total Price of the trip \\
  \hline
\end{tabular}


\subsubsection{Exported Access Programs}

\begin{tabular}{| l | l | l | l |}
  \hline
  \textbf{Name} & \textbf{In} & \textbf{Out} & \textbf{Exceptions}\\
  \hline
  render & $\mathbb{R}$ GasPrice, $\mathbb{R}$ mileage, $\mathbb{R}$ Distance, $\mathbb{R}$ TotalPrice & ReactDOM & ~\\
  \hline
\end{tabular}


\subsection{Semantics}

\subsubsection{State Variables}

None

\subsubsection{Environment Variables}

None

\subsubsection{Assumptions}

None

\subsubsection{Access Routine Semantics}

\noindent render($\mathbb{R}$ GasPrice, $\mathbb{R}$ mileage, $\mathbb{R}$ Distance, $\mathbb{R}$ TotalPrice):
\begin{itemize}
\item output: This function renders the outputs the milage, the distance traveled and the total cost.
\item exception: none
\end{itemize}

\subsubsection{Local Functions}

None

\newpage

%%%%%%%%%%%%%%%%%%%%%%%%%%%%%%%%%%%%%%%%%%%%%%%%%%%%%%%%%%%%%%%%%%%%%%%%%%%%%%%%%%%%

%%%%%%%%%%%%%%%%%%%%%%%%%%%%%%%%%%%%%%%%%%%%%%%%%%%%%%%%%%%%%%%%%%%%%%%%

\section{MIS of Navigation Module (M6)} 

\label{Module} 

\subsection{Module extends React.Component}

Navigation\_Module

\subsection{Uses}

Sidebar\_Module

\subsection{Syntax}

\subsubsection{Exported Constants}

\subsubsection{Inputs}
\begin{tabular}{| l | l | l |}
  \hline
  \textbf{Input Name} & \textbf{Type} & \textbf{Description}\\
  \hline
  evalFunc & Function & Function to evaluate the end result of the app or trip details \\
  \hline
  GasPrice & $\mathbb{R}$ & Price of gas in the start location \\
  \hline
  mileage & $\mathbb{R}$ & Mileage of car \\
  \hline
  Distance & $\mathbb{R}$  & Total distance to travel along the route in km \\
  \hline
  Total Price & $\mathbb{R}$ & Total Price of the trip \\
  \hline
\end{tabular}

\subsubsection{Exported Access Programs}

\begin{tabular}{| l | l | l | l |}
  \hline
  \textbf{Name} & \textbf{In} & \textbf{Out} & \textbf{Exceptions}\\
  \hline
  render & $\mathbb{R}$ GasPrice, $\mathbb{R}$ mileage, $\mathbb{R}$ Distance, $\mathbb{R}$ TotalPrice, Function evalFunc & ReactDOM & ~\\
  \hline
\end{tabular}

\subsection{Semantics}

\subsubsection{State Variables}

None

\subsubsection{Environment Variables}

None

\subsubsection{Assumptions}

None

\subsubsection{Access Routine Semantics}

\noindent render($\mathbb{R}$ GasPrice, $\mathbb{R}$ mileage, $\mathbb{R}$ Distance, $\mathbb{R}$ TotalPrice, Function evalFunc) :
\begin{itemize}
\item output: Renders Sidebar along with a button to minimize it.
\item exception: None
\end{itemize}

\subsubsection{Local Functions}

None

\newpage

%%%%%%%%%%%%%%%%%%%%%%%%%%%%%%%%%%%%%%%%%%%%%%%%%%%%%%%%%%%%%%%%%%%%%%%%%%%%%%%%%%%%

%%%%%%%%%%%%%%%%%%%%%%%%%%%%%%%%%%%%%%%%%%%%%%%%%%%%%%%%%%%%%%%%%%%%%%%%

\section{MIS of SideBar Module (M7)} 

\label{Module} 

\subsection{Module extends React.Component}

SideBar\_Module

\subsection{Uses}

Location\_Input\_Module, Car\_Input\_Module, Trip\_Details\_Module

\subsection{Syntax}

\subsubsection{Exported Constants}

\subsubsection{Inputs}
\begin{tabular}{| l | l | l |}
  \hline
  \textbf{Input Name} & \textbf{Type} & \textbf{Description}\\
  \hline
  evalFunc & Function & Function to evaluate the end result of the app or trip details \\
  \hline
  GasPrice & $\mathbb{R}$ & Price of gas in the start location \\
  \hline
  updateStart & Function & Update start location \\
  \hline
  updateEnd & Function & Update end location \\
  \hline
  mileage & $\mathbb{R}$ & Mileage of car \\
  \hline
  Distance & $\mathbb{R}$  & Total distance to travel along the route in km \\
  \hline
  TotalPrice & $\mathbb{R}$ & Total Price of the trip \\
  \hline
\end{tabular}

\subsubsection{Exported Access Programs}

\begin{tabular}{| l | l | l | l |}
  \hline
  \textbf{Name} & \textbf{In} & \textbf{Out} & \textbf{Exceptions}\\
  \hline
  render & $\mathbb{R}$ GasPrice, $\mathbb{R}$ mileage, $\mathbb{R}$ Distance, $\mathbb{R}$ TotalPrice, Function evalFunc, Function updateStart, Function updateEnd & ReactDOM & ~\\
  \hline
\end{tabular}

\subsection{Semantics}

\subsubsection{State Variables}

None

\subsubsection{Environment Variables}

None

\subsubsection{Assumptions}

None

\subsubsection{Access Routine Semantics}

\noindent render($\mathbb{R}$ GasPrice, $\mathbb{R}$ mileage, $\mathbb{R}$ Distance, $\mathbb{R}$ TotalPrice, Function evalFunc, Function updateStart, Function updateEnd) :
\begin{itemize}
\item output: Renders the content of Sidebar along with any other components that it uses like Trip Details Module.
\item exception: None
\end{itemize}

\subsubsection{Local Functions}

None

%%%%%%%%%%%%%%%%%%%%%%%%%%%%%%%%%%%%%%%%%%%%%%%%%%%%%%%%%%%%%%%%%%%%%%%%%%%%%%%%%%%%

%%%%%%%%%%%%%%%%%%%%%%%%%%%%%%%%%%%%%%%%%%%%%%%%%%%%%%%%%%%%%%%%%%%%%%%%

\section{MIS of Map Data Module (M7)} 

\label{Module} 

\subsection{Module extends React.Component}

Map\_Data\_Module

\subsection{Uses}

None

\subsection{Syntax}

\subsubsection{Exported Constants}

\subsubsection{Inputs}
\begin{tabular}{| l | l | l |}
  \hline
  \textbf{Input Name} & \textbf{Type} & \textbf{Description}\\
  \hline
  points & List$<\mathbb{R},\mathbb{R}>$  & coords of different points along a route \\
  \hline
\end{tabular}

\subsubsection{Exported Access Programs}

\begin{tabular}{| l | l | l | l |}
  \hline
  \textbf{Name} & \textbf{In} & \textbf{Out} & \textbf{Exceptions}\\
  \hline
  getMapData & List$<\mathbb{R},\mathbb{R}>$ points & Array$<\mathbb{R}>$ & ~\\
  \hline
\end{tabular}

\subsection{Semantics}

\subsubsection{State Variables}

None

\subsubsection{Environment Variables}

None

\subsubsection{Assumptions}

None

\subsubsection{Access Routine Semantics}

\noindent getMapData(List$<\mathbb{R},\mathbb{R}>$) :
\begin{itemize}
\item output: Array$<\mathbb{R}>$ : Retrieves all the map components used for determining the current and the destination locations.
\item exception: Map component not loaded.
\end{itemize}

\subsubsection{Local Functions}

None

\newpage

%%%%%%%%%%%%%%%%%%%%%%%%%%%%%%%%%%%%%%%%%%%%%%%%%%%%%%%%%%%%%%%%%%%%%%%%%%%%%%%%%%%%

%%%%%%%%%%%%%%%%%%%%%%%%%%%%%%%%%%%%%%%%%%%%%%%%%%%%%%%%%%%%%%%%%%%%%%%%

\section{MIS of Car Data Module (M8)} 

\label{Module} 

\subsection{Module extends React.Component}

Car\_Data\_Module

\subsection{Uses}

None

\subsection{Syntax}

\subsubsection{Exported Constants}

\subsubsection{Inputs}
\begin{tabular}{| l | l | l |}
  \hline
  \textbf{Input Name} & \textbf{Type} & \textbf{Description}\\
  \hline
  carDetails & List$<$string, string, $\mathbb{R}>$  & A car's make, model and year. \\
  \hline
\end{tabular}

\subsubsection{Exported Access Programs}

\begin{tabular}{| l | l | l | l |}
  \hline
  \textbf{Name} & \textbf{In} & \textbf{Out} & \textbf{Exceptions}\\
  \hline
  getCarData & List$<$string, string, $\mathbb{R}>$ carDetails & Array$<\mathbb{R}>$ & ~\\
  \hline
\end{tabular}

\subsection{Semantics}

\subsubsection{State Variables}

None

\subsubsection{Environment Variables}

None

\subsubsection{Assumptions}

None

\subsubsection{Access Routine Semantics}

\noindent getCarData(List$<$string, string, $\mathbb{R}>$ carDetails) :
\begin{itemize}
\item output: Array$<\mathbb{R}>$ : Retrieves all the data related to the car chosen by the user from the database, which includes the make and the model of the car, the year it was manufactured, and it's fuel mileage.
\item exception: Car component not loaded.
\end{itemize}

\subsubsection{Local Functions}

None

\newpage

%%%%%%%%%%%%%%%%%%%%%%%%%%%%%%%%%%%%%%%%%%%%%%%%%%%%%%%%%%%%%%%%%%%%%%%%%%%%%%%%%%%%

%%%%%%%%%%%%%%%%%%%%%%%%%%%%%%%%%%%%%%%%%%%%%%%%%%%%%%%%%%%%%%%%%%%%%%%%

\section{MIS of Fuel Data Module (M9)} 

\label{Module} 

\subsection{Module extends React.Component}

Fuel\_Data\_Module

\subsection{Uses}

None

\subsection{Syntax}

\subsubsection{Exported Constants}

\subsubsection{Inputs}
\begin{tabular}{| l | l | l |}
  \hline
  \textbf{Input Name} & \textbf{Type} & \textbf{Description}\\
  \hline
  coords & List$<\mathbb{R},\mathbb{R}>$  & Coordinates to find a fuel price at. \\
  \hline
\end{tabular}

\subsubsection{Exported Access Programs}

\begin{tabular}{| l | l | l | l |}
  \hline
  \textbf{Name} & \textbf{In} & \textbf{Out} & \textbf{Exceptions}\\
  \hline
  getFuelData & List$<\mathbb{R},\mathbb{R}>$ coords & Array$<\mathbb{R}>$ & ~\\
  \hline
\end{tabular}

\subsection{Semantics}

\subsubsection{State Variables}

None

\subsubsection{Environment Variables}

None

\subsubsection{Assumptions}

None

\subsubsection{Access Routine Semantics}

\noindent getFuelData(List$<\mathbb{R},\mathbb{R}>$ coords) :
\begin{itemize}
\item output: $<\mathbb{R}>$ : Retrieves fuel price for a desired location.
\item exception: Fuel component not loaded.
\end{itemize}

\subsubsection{Local Functions}

None

\newpage

%%%%%%%%%%%%%%%%%%%%%%%%%%%%%%%%%%%%%%%%%%%%%%%%%%%%%%%%%%%%%%%%%%%%%%%%%%%%%%%%%%%%

\end{document}
