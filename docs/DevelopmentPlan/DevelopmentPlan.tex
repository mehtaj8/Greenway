\documentclass{article}

\usepackage{booktabs}
\usepackage{tabularx}

\title{Development Plan\\\progname}

\author{\authname}

\date{}


\input{../Comments}
%% Common Parts

\newcommand{\progname}{Greenway} % PUT YOUR PROGRAM NAME HERE
\newcommand{\authname}{Team \#11, Roadkill
\\ Priyansh Shah, shahp36
\\ Utsharga Rozario, rozariou
\\ Jash Mehta, mehtaj8
\\ Bilal Shaikh, shaikb2
\\ Pranay Kotian, kotianp
\\ Sharjil Mohsin, mohsis2} % AUTHOR NAMES                  

\usepackage{hyperref}
    \hypersetup{colorlinks=true, linkcolor=blue, citecolor=blue, filecolor=blue,
                urlcolor=blue, unicode=false}
    \urlstyle{same}
                                


\begin{document}

\begin{table}[!hbp]
    \caption{Revision History} \label{RevisionHistory}
    \begin{tabularx}{\textwidth}{llX}
        \toprule
            \textbf{Date} & \textbf{Developer(s)} & \textbf{Change}\\
        \midrule
            Sep 25, 2022 & 
            \begin{tabular}{@{}c@{}}Priyansh, Utsharga, Sharjil,\\Jash, Bilal, Pranay\end{tabular}
            & Rev. 0\\            
            ... & ... & ...\\
        \bottomrule
    \end{tabularx}
\end{table}


\maketitle

This document outlines the development plan for our capstone project Greenway. Greenway will calculate fuel consumption and estimated gas cost for different routes, showing the user the most optimal route for reduced fuel consumption. It will do so using the Google Maps API as well as a database of car fuel consumption.

\section{Team Meeting Plan}
The team will meet once a week either Friday evening or Sunday afternoon to work 
on the development of Greenway. Each member will contribute equally in discussions 
centered around project design and development. The purpose of each meeting is to 
come together to make progress on the next milestone/deliverable for the project. 
Detailed meeting agenda points and to-do items will be specified before each meeting. 
A member of the team will take meeting minutes and record our discussion in our 
group Notion. Each meeting will end with a discussion of what needs to get completed 
before our next meeting.

\section{Team Communication Plan}
The team will communicate with one another primarily using Facebook messenger group 
chat. Issues regarding specifics within the source code or documentation will be 
raised on GitHub when necessary. Secondary documentation of our projects and 
conversations are recorded on the group Notion page. For urgent matters, all of 
our personal phone numbers and emails are also accessible.

\section{Team Member Roles}
There is no specific leader, we each have our own roles and responsibilities.

\subsection*{Jash Mehta}

\begin{itemize}
	\item Developer
	\item Project Manager
  \item Specialization in Git Version Control 
\end{itemize}

\subsection*{Utsharga Rozario}

\begin{itemize}
	\item Developer
	\item Communication Officer
  \item Specialization in UI/UX and Frontend Development
\end{itemize}

\subsection*{Sharjil Mohsin}

\begin{itemize}
	\item Developer
	\item Minutes Taker
  \item Specialization in Backend Development
\end{itemize}

\subsection*{Priyansh Shah}

\begin{itemize}
	\item Developer
  \item Specialization in LaTeX and Documentation
\end{itemize}

\subsection*{Bilal Shaikh}

\begin{itemize}
	\item Developer
	\item Specialization in mapping APIs
\end{itemize}

\subsection*{Pranay Kotian}

\begin{itemize}
	\item Developer
	\item Scrum Master
  \item Specialization in Mathematical Calculations
\end{itemize}

\section{Workflow Plan}
Our team’s workflow will begin with members of the team being assigned a feature of the product or a bug to fix based on the team meetings we had. We will use GitHub Issues to assign a member of a team a feature or a bug, and make sure to link the issue with the branch they will be working with.  Before a developer starts working on a feature assigned to them or fixing a bug, they will pull the latest changes from the main branch and then create a new branch from it which will contain the addition of new or changed code. The developer will then commit the changes to this new branch with a commit message outlining what changes have been made, before creating a pull request when these changes are ready for review. Once a pull request is opened, GitHub Actions will run automated tests and unless all the tests have passed, the pull request will be blocked from being merged. After all the tests have passed and one other developer approves the changes being made, the changes will be ready to merge into the main branch. All these steps will be performed by all the developers of this team for the duration of the whole project.
\\
\\
The full workflow is summarized below:
\begin{enumerate}
	\item Create and assign issues on the GitHub Issues interface
  \begin{enumerate}
    \item Fill in the description for the issue
  \end{enumerate}
	\item Pull changes from the main branch
  \item Create a new branch off of the main branch
  \item Commit changes to this branch
  \item Create a pull request for the changes to the new branch
  \begin{enumerate}
    \item Create a pull request for the changes to the new branch
    \item Add the corresponding description for the PR and link to the issue
    \item Have one other member of the team approves the changes
  \end{enumerate}
  \item Merge the changes into the main branch
  \item Once the changes are merged, remove the branch and issue
\end{enumerate}

\section{Proof of Concept Demonstration Plan}

The main risks associated with POC demo is concerned with are web services which the project will 
depend on. One would be the API being used for MAPS services; another one would be a possible database 
that the app will need to connect to. We will demonstrate that the services our app depends on have a 
high enough uptime that function of our app will not be affected. Additionally, we will design our app 
to be ready for any possible service being down so the user will be affected by a buggy application. Lastly, 
if primary functions are inaccessible during app downtime the user will be notified and the app will present 
alternate solutions or lock the user out of the app until functionality is restored. Testing for the backend 
of the application will be relatively easy since there will expected inputs and outputs for the application. 
But the frontend testing will be a relatively hard as good coverage is hard to achieve due to the number of 
different ways a user can interact with the website. 

\section{Technology}

\begin{itemize}
\item React.js: This will be the framework that will be used for the frontend web development.
\item Node.js: This will be the framework that will be used for the backend development.
\item MongoDB: This will be the database used for storing information for the database.
\item VSCode: This will be used for frontend and backend development of the application.
\item GitHub: This will be used for version control and tracking issues within code for the application.
\end{itemize}

\section{Coding Standard}
The main coding standard that we will be following for the front end is Airbnb React/JSX Style. And this standard will be dictating all of our React code and similarly JavaScript code. So all of our code for Node.js and React.js will be under this standard. This standard will also cover the code written to integrate MongoDB as that will be in JavaScript in the Node.js application.

\section{Project Scheduling}
This following link contains detailed information about our project scheduling and our Gantt Charts:\\
\url{https://github.com/mehtaj8/Greenway/tree/main/docs/DevelopmentPlan}

\end{document}