\documentclass[12pt, titlepage]{article}

\usepackage{booktabs}
\usepackage{tabularx}
\usepackage{hyperref}
\hypersetup{
    colorlinks,
    citecolor=black,
    filecolor=black,
    linkcolor=red,
    urlcolor=blue
}
\usepackage[round]{natbib}

\input{../Comments}

\begin{document}

\title{Test Report: Project Title} 
\author{Author Name}
\date{\today}
	
\maketitle

\pagenumbering{roman}

\section{Revision History}

\begin{tabularx}{\textwidth}{p{3cm}p{2cm}X}
\toprule {\bf Date} & {\bf Version} & {\bf Notes}\\
\midrule
Date 1 & 1.0 & Notes\\
Date 2 & 1.1 & Notes\\
\bottomrule
\end{tabularx}

~\newpage

\section{Symbols, Abbreviations and Acronyms}

\renewcommand{\arraystretch}{1.2}
\begin{tabular}{l l} 
  \toprule		
  \textbf{symbol} & \textbf{description}\\
  \midrule 
  T & Test\\
  \bottomrule
\end{tabular}\\

\wss{symbols, abbreviations or acronyms -- you can reference the SRS tables if needed}

\newpage

\tableofcontents

\listoftables %if appropriate

\listoffigures %if appropriate

\newpage

\pagenumbering{arabic}

This document ...

\section{Functional Requirements Evaluation}
Tests taken from V and V Plan section 5.1 are evaluated here and the corresponding section names are used as well.

\subsection{Start Screen}

Start screen covers functional requirements 1-5 from the SRS as all those requirements 
are related to the inputs and functionality that exists in the program before any 
processing or output is shown
		
\paragraph{Preliminary Information Tests} 

\begin{enumerate}

\item{preliminary-information-test-1\\}

Control: Automatic
					
Initial State: No input in the start screen
					
Input: Start Location
					
Desired output: Status Message stating if the start Location was accepted

Actual Output: No status message found

Result: Test case failed as the app failed to display a popup of location being validated.

Impact: The app will assume the use of valid start location as input as the core use of this app is 
to find the cost of trips not validate locations it is given.
					
\item{preliminary-information-test-2\\}

Control: Automatic
					
Initial State: No input in the start screen
					
Input: Location of the Destination
					
Output: Status Message stating if the Location was accepted

Desired output: Status Message stating if the destination location was accepted

Actual Output: No status message found

Result: Test case failed as the app failed to display a popup of the location being validated.

Impact: The app will assume the use of valid end locations as input as the core use of this app is 
to find the cost of trips not validate locations it is given.

\item{preliminary-information-test-3\\}

Control: Automatic
					
Initial State: No input in the start screen
					
Input: Car Details

Desired output: Status Message stating if the Car Information was accepted

Actual Output: No status message found

Result: Test case failed as the app failed to display a popup of the status message validating car details.

Impact: The app with its dropdown menus only allows the user to select valid cars and similarily only asks for 
mileage as alternative.

\item{preliminary-information-test-4\\}

Control: Automatic
					
Initial State: Accepted Car details in the start screen
					
Input: Car mileage/fuel economy information

Desired output: Status Message stating if the Car mileage/fuel economy information was accepted

Actual Output: No status message found

Result: Test case failed as the app failed to display a popup of the status message validating mileage information.

Impact: Small popup known as toast will be added to the app to indicate a valid/invalid value has been added to the mileage.

\item{preliminary-information-test-5\\}

Control: Automatic
					
Initial State: Accepted Car details in the start screen
					
Input: Car information

Desired output: Status Message stating if the Car information was updated

Actual Output: No status message found

Result: Test case failed as the app failed to display a popup of the status message stating if the Car information was updated successfully.

Impact: The popup does not need to exist as the app should require a full reset as a lot of different information needs to be 
updated to make this feature viable for the major goal of this app.

\end{enumerate}

\subsection{Map Interactions}

Map Interaction covers functional requirements 6-8 as these requirements 
are concerned with functionality related to the map on display.
		
\paragraph{Map Tests}

\begin{enumerate}

\item{map-test-1\\}

Control: Manual
					
Initial State: Start Screen finished
					
Input: None

Desired output: Map displaying a route from start to end based on start screen input

Actual Output: Found full route displayed on the map to indicate the route the app considers most ideal

Result: Test case passed as the desired output was found in the app.

Impact: None
					
\item{map-test-2\\}

Control: Manual
					
Initial State: Start Screen finished
					
Input: None

Desired output: Map displaying all gas stations that it can possibly encounter along the route from a start to end destination

Actual Output: Found no gas stations along the route being displayed.

Result: Test case failed as the desired output was not found in the app.

Impact: The requirement that was used in the derivation process for this is not important anymore as the app functions on the basis that refueling 
is done at the start of the trip.
					
\item{map-test-3\\}

Control: Manual
					
Initial State: Start Screen finished
					
Input: None

Desired output: Map displaying all gas station prices as they come up along the displayed route

Actual Output: Found no gas stations as such  along the route being displayed.

Result: Test case failed as the desired output was not found in the app.

Impact: Gas prices could not feasibly be found for all the gas stations as such the requirements used to derive this test case are not 
valid anymore.

\end{enumerate}

\subsection{Backend Processing}

Backend processing covers functional requirements 9-12 as these 
requirements are concerned with calculations and, external data collection and processing 
alike.
		
\paragraph{Backend Tests}

\begin{enumerate}

\item{backend-test-1\\}

Control: Automatic
					
Initial State: Start Screen finished
					
Input: Route Details

Desired output: Route returned to the user is one that is optimizing the distance and elevation perfectly to reduce fuel costs

Actual Output: Route returned to the user is optimized to minimize fuel costs

Result: Test case passed as the main goal of the optimization is achieved as minimal fuel costs are being shown.

Impact: None.
					
\item{backend-test-2\\}

Control: Automatic
					
Initial State: Start Screen finished
					
Input: Route Details

Desired output: The database call on the backend returns correct elevation data for a route.

Actual Output: The route details for multiple routes was matched with correct reference data in the framework and was found to be identical.

Result: Test case passed as the desired output was achieved.

Impact: None.
					
\item{backend-test-3\\}

Control: Automatic
					
Initial State: Start Screen finished
					
Input: Route Details

Desired output: Correct mileage is given by backend for a certain coordinate.

Actual Output: Mileage that matched reference value is given by the backend.

Result: Test case passed as the desired output was achieved.

Impact: None.
					
\item{backend-test-4\\}

Control: Automatic
					
Initial State: Start Screen finished
					
Input: Route Details

Desired output: Correct total cost is calculated by the backend for a certain route with a start and end decisions as the route details.

Actual Output: All data required for correct total cost is given by the backend which is verified with reference data and equated in the front end.

Result: Test case passed as the desired output was achieved.

Impact: None.

\end{enumerate}

\section{Nonfunctional Requirements Evaluation}

\subsection{Usability}
		
\subsection{Performance}

\subsection{etc.}
	
\section{Comparison to Existing Implementation}	

This section will not be appropriate for every project.

\section{Unit Testing}

\section{Changes Due to Testing}

\subsection{Preliminary Information Tests}
To start with there will be some changes incoming to the SRS firstly FR-1 and FR-2 will be removed or modified due to 
preliminary-information-test-1 and 2. And FR-3 will be modified due to the results of test preliminary-information-test-3. 
The test preliminary-information-test-4 showed there is a need for small popup with the use of a toast ui component in the 
app and this will be added to the application. The test preliminary-information-test-5 showed that the SRS needs to be updated 
with removal or modification of FR-5 due to the changed priorities of the application.

\subsection{Map Interactions}
The test Mapmap-test-2 showed that the SRS needs to be changed for FR-7 as the refueling functionality has been modified from 
original concept of the application so this requirement will be changed or removed entirely. Similar idea for map-test-3 as the 
refueling functionality of the original concept has changed there is no need to show individual fuel prices in a certain area 
with the need of it being removed/changed.

\subsection{Backend Processing}
No changes to the backend from the testing.

\section{Automated Testing}
		
\section{Trace to Requirements}
		
\section{Trace to Modules}		

\section{Code Coverage Metrics}

\bibliographystyle{plainnat}
\bibliography{../../refs/References}

\newpage{}
\section*{Appendix --- Reflection}

The information in this section will be used to evaluate the team members on the
graduate attribute of Lifelong Learning.  Please answer the following questions:

\begin{enumerate}
  \item 
  \item 
\end{enumerate}

\end{document}