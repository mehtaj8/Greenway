\documentclass{article}

\usepackage{tabularx}
\usepackage{booktabs}

\title{Problem Statement and Goals\\\progname}

\author{\authname}

\date{}

%% Comments

\usepackage{color}

\newif\ifcomments\commentstrue %displays comments
%\newif\ifcomments\commentsfalse %so that comments do not display

\ifcomments
\newcommand{\authornote}[3]{\textcolor{#1}{[#3 ---#2]}}
\newcommand{\todo}[1]{\textcolor{red}{[TODO: #1]}}
\else
\newcommand{\authornote}[3]{}
\newcommand{\todo}[1]{}
\fi

\newcommand{\wss}[1]{\authornote{blue}{SS}{#1}} 
\newcommand{\plt}[1]{\authornote{magenta}{TPLT}{#1}} %For explanation of the template
\newcommand{\an}[1]{\authornote{cyan}{Author}{#1}}

%% Common Parts

\newcommand{\progname}{ProgName} % PUT YOUR PROGRAM NAME HERE
\newcommand{\authname}{Team \#, Team Name
\\ Student 1 name and macid
\\ Student 2 name and macid
\\ Student 3 name and macid
\\ Student 4 name and macid} % AUTHOR NAMES                  

\usepackage{hyperref}
    \hypersetup{colorlinks=true, linkcolor=blue, citecolor=blue, filecolor=blue,
                urlcolor=blue, unicode=false}
    \urlstyle{same}
                                


\begin{document}

\maketitle

\begin{table}[!hbp]
    \caption{Revision History} \label{RevisionHistory}
    \begin{tabularx}{\textwidth}{llX}
        \toprule
            \textbf{Date} & \textbf{Developer(s)} & \textbf{Change}\\
        \midrule
            Sep 25, 2022 & 
            \begin{tabular}{@{}c@{}}Priyansh, Utsharga, Sharjil,\\Jash, Bilal, Pranay\end{tabular}
            & Rev. 0\\            
            ... & ... & ...\\
        \bottomrule
    \end{tabularx}
\end{table}

\newpage

\section{Problem Statement}

\subsection{Problem}
Navigation applications are commonly used applications while driving to get directions 
from point A to B, but these applications never tell you how much it costs you 
to get there or how much gas was used on the trip. When carpooling with friends, 
the driver of the vehicle always asks everyone in the group for gas money and 
often times these calculations are mere estimates that are not always very 
accurate. People often wonder how much it costs to get to a destination before 
starting your journey, having an application perform these cost and fuel calculations 
based on real time gas pricing information ensures you save the most money on your 
journey while minimizing gas usage to encourage a more sustainable lifestyle.

\subsection{Inputs and Outputs}
\begin{tabular}{| p{0.5\linewidth} | p{0.5\linewidth} |}
    \hline
    \textbf{Inputs} & \textbf{Outputs}\\ \hline
    Starting location & Most fuel efficient route\\ \hline
    Ending location & Cost of driving (from starting to ending location)\\ \hline
    Car specifications (year, make, model, trim) or fuel economy & Suggested stops at gas stations/supercharger locations based on real-time prices\\ \hline
    User estimate of how much fuel or distance worth of fuel is currently in the gas tank. & \\ \hline
\end{tabular}

\subsection{Stakeholders}
\begin{itemize}
    \item Product Users - The product users include anyone who will be driving and hence will 
    be using the product. The product users have an influence on the requirements of the application 
    and its overall development. 
    \item Development Team - The development team is the group of people who are developing the application. 
    They include, but are not limited to, the software developers who develop the application and ensure easy 
    to use user experience.
    \item Teaching staff/Instructors - The teaching staff and instructors will be directly interacting with 
    our product and assessing its usability and user experience. 
\end{itemize}

\subsection{Environment}
The environment of the web application will be determined by the users/stakeholders, i.e., any personal computers (e.g. Windows, MacBook, Linux) or mobile devices, and compatible browsers (e.g. Google Chrome, Microsoft Edge, etc.) 

\section{Goals}
    \begin{tabular}{| p{0.5\linewidth} | p{0.5\linewidth} |}
		\hline
\textbf{Goals} & \textbf{Importance}\\ \hline
The finished product is able to determine the fuel economy of any given car as input. & This is one of the fundamental functions which will help the product appeal to the stakeholders. The larger the database, the more it might be used.\\ \hline
The finished product is able to calculate the trip cost based on mileage and distance travelled, while factoring in elements like the terrain. & This will ensure trips costs are calculated as accurately as possible factoring in trip length, fluctuating gas prices, and elevation changes over the route.\\ \hline 
This will ensure trips costs are calculated as accurately as possible factoring in trip length, fluctuating gas prices, and elevation changes over the route. & This allows the product to be used reliable for long-term trips without need to regularly updated. Limiting the number of updates will also ensure product reliability and user familiarity.\\ \hline 
This allows the product to be used reliable for long-term trips without need to regularly updated. Limiting the number of updates will also ensure product reliability and user familiarity. & This will ensure the user gets gas prices in real time, hence cost of trip updates with changing prices.\\ \hline
This will ensure the user gets gas prices in real time, hence cost of trip updates with changing prices. & This allows the product to be robust and resistant to being outdated. \\ \hline  
    \end{tabular}

\section{Stretch Goals}
    \begin{tabular}{| p{0.3\linewidth} | p{0.7\linewidth} |}
    \hline
\textbf{Goals} & \textbf{Importance}\\ \hline
The finished product is portable. & Although our product will be a web application, having it available as a mobile application will allow users to know the gas mileage on the go instead of having to see it on a computer.\\ \hline
The finished product can interact with a voice assistant. & Currently, Greenway is only accessible to the user by opening or entering the application. By letting the user use a voice assistant, they will be able to get these cost and fuel calculations from outside the application.\\ \hline 
The finished product can support navigation with Google Maps. & Our current goal only lets Greenway serve as a trip planner rather than an actual application you can use to navigate with while you drive. Allowing users to navigate with it during the trip lets users not have to remember the route and stops to enter into their navigation application of choice beforehand.\\ \hline 
\end{tabular}   

\end{document}